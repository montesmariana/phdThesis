% Options for packages loaded elsewhere
\PassOptionsToPackage{unicode}{hyperref}
\PassOptionsToPackage{hyphens}{url}
%
\documentclass[
]{book}
\usepackage{lmodern}
\usepackage{amsmath}
\usepackage{ifxetex,ifluatex}
\ifnum 0\ifxetex 1\fi\ifluatex 1\fi=0 % if pdftex
  \usepackage[T1]{fontenc}
  \usepackage[utf8]{inputenc}
  \usepackage{textcomp} % provide euro and other symbols
  \usepackage{amssymb}
\else % if luatex or xetex
  \usepackage{unicode-math}
  \defaultfontfeatures{Scale=MatchLowercase}
  \defaultfontfeatures[\rmfamily]{Ligatures=TeX,Scale=1}
\fi
% Use upquote if available, for straight quotes in verbatim environments
\IfFileExists{upquote.sty}{\usepackage{upquote}}{}
\IfFileExists{microtype.sty}{% use microtype if available
  \usepackage[]{microtype}
  \UseMicrotypeSet[protrusion]{basicmath} % disable protrusion for tt fonts
}{}
\makeatletter
\@ifundefined{KOMAClassName}{% if non-KOMA class
  \IfFileExists{parskip.sty}{%
    \usepackage{parskip}
  }{% else
    \setlength{\parindent}{0pt}
    \setlength{\parskip}{6pt plus 2pt minus 1pt}}
}{% if KOMA class
  \KOMAoptions{parskip=half}}
\makeatother
\usepackage{xcolor}
\IfFileExists{xurl.sty}{\usepackage{xurl}}{} % add URL line breaks if available
\IfFileExists{bookmark.sty}{\usepackage{bookmark}}{\usepackage{hyperref}}
\hypersetup{
  pdftitle={Cloudspotting: Visual analytics for distributional semantics},
  pdfauthor={Mariana Montes},
  hidelinks,
  pdfcreator={LaTeX via pandoc}}
\urlstyle{same} % disable monospaced font for URLs
\usepackage{color}
\usepackage{fancyvrb}
\newcommand{\VerbBar}{|}
\newcommand{\VERB}{\Verb[commandchars=\\\{\}]}
\DefineVerbatimEnvironment{Highlighting}{Verbatim}{commandchars=\\\{\}}
% Add ',fontsize=\small' for more characters per line
\usepackage{framed}
\definecolor{shadecolor}{RGB}{248,248,248}
\newenvironment{Shaded}{\begin{snugshade}}{\end{snugshade}}
\newcommand{\AlertTok}[1]{\textcolor[rgb]{0.94,0.16,0.16}{#1}}
\newcommand{\AnnotationTok}[1]{\textcolor[rgb]{0.56,0.35,0.01}{\textbf{\textit{#1}}}}
\newcommand{\AttributeTok}[1]{\textcolor[rgb]{0.77,0.63,0.00}{#1}}
\newcommand{\BaseNTok}[1]{\textcolor[rgb]{0.00,0.00,0.81}{#1}}
\newcommand{\BuiltInTok}[1]{#1}
\newcommand{\CharTok}[1]{\textcolor[rgb]{0.31,0.60,0.02}{#1}}
\newcommand{\CommentTok}[1]{\textcolor[rgb]{0.56,0.35,0.01}{\textit{#1}}}
\newcommand{\CommentVarTok}[1]{\textcolor[rgb]{0.56,0.35,0.01}{\textbf{\textit{#1}}}}
\newcommand{\ConstantTok}[1]{\textcolor[rgb]{0.00,0.00,0.00}{#1}}
\newcommand{\ControlFlowTok}[1]{\textcolor[rgb]{0.13,0.29,0.53}{\textbf{#1}}}
\newcommand{\DataTypeTok}[1]{\textcolor[rgb]{0.13,0.29,0.53}{#1}}
\newcommand{\DecValTok}[1]{\textcolor[rgb]{0.00,0.00,0.81}{#1}}
\newcommand{\DocumentationTok}[1]{\textcolor[rgb]{0.56,0.35,0.01}{\textbf{\textit{#1}}}}
\newcommand{\ErrorTok}[1]{\textcolor[rgb]{0.64,0.00,0.00}{\textbf{#1}}}
\newcommand{\ExtensionTok}[1]{#1}
\newcommand{\FloatTok}[1]{\textcolor[rgb]{0.00,0.00,0.81}{#1}}
\newcommand{\FunctionTok}[1]{\textcolor[rgb]{0.00,0.00,0.00}{#1}}
\newcommand{\ImportTok}[1]{#1}
\newcommand{\InformationTok}[1]{\textcolor[rgb]{0.56,0.35,0.01}{\textbf{\textit{#1}}}}
\newcommand{\KeywordTok}[1]{\textcolor[rgb]{0.13,0.29,0.53}{\textbf{#1}}}
\newcommand{\NormalTok}[1]{#1}
\newcommand{\OperatorTok}[1]{\textcolor[rgb]{0.81,0.36,0.00}{\textbf{#1}}}
\newcommand{\OtherTok}[1]{\textcolor[rgb]{0.56,0.35,0.01}{#1}}
\newcommand{\PreprocessorTok}[1]{\textcolor[rgb]{0.56,0.35,0.01}{\textit{#1}}}
\newcommand{\RegionMarkerTok}[1]{#1}
\newcommand{\SpecialCharTok}[1]{\textcolor[rgb]{0.00,0.00,0.00}{#1}}
\newcommand{\SpecialStringTok}[1]{\textcolor[rgb]{0.31,0.60,0.02}{#1}}
\newcommand{\StringTok}[1]{\textcolor[rgb]{0.31,0.60,0.02}{#1}}
\newcommand{\VariableTok}[1]{\textcolor[rgb]{0.00,0.00,0.00}{#1}}
\newcommand{\VerbatimStringTok}[1]{\textcolor[rgb]{0.31,0.60,0.02}{#1}}
\newcommand{\WarningTok}[1]{\textcolor[rgb]{0.56,0.35,0.01}{\textbf{\textit{#1}}}}
\usepackage{longtable,booktabs}
\usepackage{calc} % for calculating minipage widths
% Correct order of tables after \paragraph or \subparagraph
\usepackage{etoolbox}
\makeatletter
\patchcmd\longtable{\par}{\if@noskipsec\mbox{}\fi\par}{}{}
\makeatother
% Allow footnotes in longtable head/foot
\IfFileExists{footnotehyper.sty}{\usepackage{footnotehyper}}{\usepackage{footnote}}
\makesavenoteenv{longtable}
\usepackage{graphicx}
\makeatletter
\def\maxwidth{\ifdim\Gin@nat@width>\linewidth\linewidth\else\Gin@nat@width\fi}
\def\maxheight{\ifdim\Gin@nat@height>\textheight\textheight\else\Gin@nat@height\fi}
\makeatother
% Scale images if necessary, so that they will not overflow the page
% margins by default, and it is still possible to overwrite the defaults
% using explicit options in \includegraphics[width, height, ...]{}
\setkeys{Gin}{width=\maxwidth,height=\maxheight,keepaspectratio}
% Set default figure placement to htbp
\makeatletter
\def\fps@figure{htbp}
\makeatother
\setlength{\emergencystretch}{3em} % prevent overfull lines
\providecommand{\tightlist}{%
  \setlength{\itemsep}{0pt}\setlength{\parskip}{0pt}}
\setcounter{secnumdepth}{5}
\usepackage{booktabs}
\usepackage{fontspec}
\usepackage{multirow}
\usepackage{multicol}
\usepackage{colortbl}
\usepackage{hhline}
\usepackage{longtable}
\usepackage{array}
\usepackage{hyperref}
\ifluatex
  \usepackage{selnolig}  % disable illegal ligatures
\fi
\usepackage[style=apa,]{biblatex}
\addbibresource{assets/bib/PhDCitations.bib}
\addbibresource{assets/bib/packages.bib}

\title{Cloudspotting: Visual analytics for distributional semantics}
\author{Mariana Montes}
\date{2021-05-01}

\begin{document}
\maketitle

{
\setcounter{tocdepth}{1}
\tableofcontents
}
\begin{Shaded}
\begin{Highlighting}[]
\NormalTok{knitr}\SpecialCharTok{::}\NormalTok{opts\_chunk}\SpecialCharTok{$}\FunctionTok{set}\NormalTok{(}\AttributeTok{echo =} \ConstantTok{FALSE}\NormalTok{, }\AttributeTok{message =} \ConstantTok{FALSE}\NormalTok{)}
\end{Highlighting}
\end{Shaded}

\hypertarget{introduction}{%
\chapter*{Introduction}\label{introduction}}
\addcontentsline{toc}{chapter}{Introduction}

Here I'm starting the first draft ever of my PhD dissertation.
There are reports scattered all over the place, but here I will try to write things already
thinking of a Final Product. I need a layout (briefly discussed at the end of February) and
I will slowly start building the final text.

The original title (in any case, the title of my project) was
\emph{Methodological triangulation in corpus-based distributional semantics},
but the triangulation part was lost somewhere along the way.

What I really did do was develop the visualization,
based on \href{https://github.com/tokenclouds/tokenclouds.github.io}{Thomas Wielfaert's original code},
analyze the parameter settings in multiple ways,
checking different combinations and exploring different avenues,
and lead, check, study and compare manual annotation of 32 lemmas.

That is what I'm going to write about.

\hypertarget{acknowledgments}{%
\chapter*{Acknowledgments}\label{acknowledgments}}
\addcontentsline{toc}{chapter}{Acknowledgments}

The words in this pages, the thoughts they try to convey, are the result of
years of thinking, discussing, studying, learning. My voice weaves them together,
but it draws from so many sources that have encouraged my growth, stood by me,
fed my curiosity, passion and enthusiasm for everything that makes up this text.

For their support and their ideas, I want to thank my supervisors Dirk Geeraerts,
Dirk Speelman and Benedikt Szmrecsanyi. Each of them gave their own piece, building
my confidence, encouraging me to keep exploring and learning. My main supervisor,
Dirk Geeraerts, deserves a special acknowledgment for all the hours in deep
discussion on the complex issue that is \emph{meaning}, and what all this is about after
all. I appreciate his patience and his engagement. Every time we talked I left
feeling more excited and passionate about the topic, more confident and happy.
Hartelijk bedankt.

I must also thank my colleagues from the Linguistics Department at KU Leuven,
which at the different stages of my years here have been a lovely company and
support system. In particular, I would like to thank the members of the
Nephological Semantics project, with whom I shared so much of the excitement and
frustrations of our common project.

When I joined the project, I already brought with me my own history and connection
to (cognitive) linguistics, corpora, statistics and programming. I honestly wouldn't
be where I am today if it weren't for my parents, Miguel and Patricia. They have
always supported and fostered my study and my interests, given me the tools to
grow, to face new challenges. I know I can, because they believe it too.

\hypertarget{part-visualization-tool}{%
\part{Visualization tool}\label{part-visualization-tool}}

\hypertarget{an-interface-to-the-world-of-clouds}{%
\chapter{An interface to the world of clouds}\label{an-interface-to-the-world-of-clouds}}

In this part (which will have who knows how many chapters) I mean to describe
the visualization tool. It was originally created by \href{https://github.com/tokenclouds/tokenclouds.github.io}{Thomas Wielfart},
but around July 2019 I started to play around with the code and learn Javascript
and \href{https://d3js.org}{D3}, culminating in the \href{https://github.com/qlvl/NephoVis}{present version} \autocite{montes.qlvl_2021}.

This section will include the technical description of the workflow, as it pertains
to the tool itself and to the processing work made before (the Python module, other
Python and R functions), and a sort of manual of how it's used. It will be more or less
redundant with the paper I wrote with Kris in December \autocite{montes.heylen_Submitted},
more or less like vignettes for documentation (as of now, it is still not documented).

The high level of automatization of vector space models makes it possible to analyze semantic patterns in huge amounts of corpus data. While this has the advantage of making semantic analyses more data-driven, this also implies a lower level of control for the linguist, who is confronted with advanced statistical output that is not straightforwardly interpretable in terms of the linguistic phenomena under investigation. In the case of vector space models, and especially an approach that combines multiple parameter settings to generate a multitude of models, the output is an array of distance matrices that ``somehow'' models different aspects of probabilistic semantic structure. To interpret and understand statistical modelling from a linguistic perspective, we turn to visual analytics.

Visual analytics aims to integrate statistical data analysis with techniques from information visualization so that human analysts can recognize, interpret and reason about the statistical patterns that the data analysis reveals \autocite{card.etal_1999}. Importantly, a visual analytics approach offers a manipulable, interactive visualization that, unlike static diagrams, enables the exploration of a space of parameter values and modelling outputs.

Lexical semantic research with vector space models would benefit from a tool that aids
i) the visualization of distance matrices based on high-dimensional data,
ii) the comparison of multiple models and
iii) more detailed examination of the input to these models.

\hypertarget{parameter-settings}{%
\chapter{Parameter settings}\label{parameter-settings}}

In this chapter I will describe the various parameter settings we have explored:
which are the possible decisions, which ones we have set and which were looked at,
why. This should be preceded by an explanation of the workflow itself.

\hypertarget{first-steps}{%
\section{First steps}\label{first-steps}}

Both the targets and the first and second order features are lemma/part-of-speech pairs,
such as \emph{haak/verb} (the verb \emph{haken} `to hook/crochet'),
\emph{beslissing/noun} (the noun \emph{beslissing} `decision'), \emph{in/prep}
(the preposition \emph{in} `in').
The features (or context words) can have any part of speech except for punctuation
and have a minimum relative frequency frequency of 1 in 2 million (absolute frequency of 227)
after discarding punctuation from the token count in the full
QLVLNews corpus. There are 60533 such lemmas in the corpus.

(This threshold is more or less arbitrary, but we're assuming that words with a lower frequency
won't have a rich enough vectorial representation.)

In the steps between defining corpus and types and obtaining a the token-level vectors,
we have two main kinds of parameters to explore.
\textbf{First-order parameters} influence which context features will be selected
from the immediate environment of the target tokens,
while the \textbf{second-order parameters} influence the shape of the vectors
that represent such first-order features.

In order to visualize the tokens, we have performed dimensionality reduction, i.e.
a process by which we try to represent relative distances between items in a low-dimensional space
while preserving the distances in high-dimensional space as much as possible.
This procedure will be described in {[}appropriate section{]}.

\hypertarget{first-order-selection-parameters}{%
\section{First-order selection parameters}\label{first-order-selection-parameters}}

We call the immediate context of a token the \textbf{first order context}: therefore,
first-order parameters are those that influence which elements in the immediate environment
of the token will be included in modeling said token. This was made in two stages:
one dependent on whether syntactic information was use, and one independent of it.

It goes without saying that the parameter space is virtually unlimited, and decisions
had to be made regarding which particular settings would be explored. We tried to
keep the parameter settings different enough from each other to have some variation.
The decisions were based on a mix of literature \autocite{kiela.clark_2014},
linguistic intuition and generalizations over the annotation of our very targets.
As part of the annotation task, the annotators had to select the items in the
immediate context that had helped them select the appropriate tag. In order to remove
noise from misunderstandings and idiosyncrasy, we only looked at pairs (or trios) of
annotators that had agreed with each other and with our final annotation and ranked
the context words over which they had agreed. The distance and dependency information
of these context words were used to inform some of our decisions below.

On the first stage, the main distinction is made by \texttt{BASE}: between bag-of-words (\texttt{BOW}) based
and dependency-based models (\texttt{LEMMAPATH} and \texttt{LEMMAREL}).
The former are further split by window size (\texttt{FOC-WIN}), part-of-speech filters (\texttt{FOC-POS})
and whether sentence boundaries are respected (\texttt{BOUND}).

\begin{description}
\tightlist
\item[\texttt{FOC-WIN} (first order window)]
A symmetric window of 3, 5 or 10 tokens to each side of the target was used.

Of course, virtually any other value is possible {[}add references!{]}. Windows of
5 and 10 are typical in the literature {[}sources?{]}, while 3 was enough to capture most
of the context words tagged as informative by the annotators.
\item[\texttt{FOC-POS} (first order part-of-speech)]
A restriction was placed to only select (common) nouns, adjectives, verbs and adverbs (\texttt{lex})
in the surroundings of the token. If no restriction is placed, the value of this parameter is \texttt{all}.

Of course, other selections are possible. {[}add reference{]} distinguish between
\texttt{nav}, which only includes common nouns, adjectives, and verbs, and \texttt{nav-nap}, which
expand the selection to proper nouns, adverbs and prepositions.

A more detailed research on different combinations would be material for further
research. As we will see, the \texttt{lex} filter is often redundant with the one based on
association strength.
\item[\texttt{BOUNDARIES}]
Given information on the limits of sentences (e.g.~in corpora annotated for
syntactic dependencies), we can exclude context words beyond the sentence of the target (\texttt{bound})
or include them (\texttt{nobound}).

This parameter seems to be virtually irrelevant. It was thought as a way of
leveling the comparison with the dependency-based models, which by definition don't
include context words beyond the sentence, but they don't seem to make a difference.
\end{description}

The distinction between BOW- and dependency-based model doesn't rely so much on
which context words are selected but on how tailored the selection is to the specific
tokens. For example, a closed-class element like a preposition may be distinctive
of particular usage patterns in which a term might occur. However, such a frequent,
multifunctional word could easily occur in the immediate raw context of the target
without actually being related to it. Unfortunately, just narrowing the window span
doesn't solve the problem, since it would also drastically reduce the number of
context words available for the token and for any other token in the model.
In contrast, we could also have context words that are directly linked to the target
but separated by many other words in between, and enlarging the window to include
them would imply too much noise for this token and for any other token in the model.

A dependency-based model, instead, will only include context words in a certain
syntactic relationship to the target, regardless of the number of words in between.
The actual selection process takes two forms in our case: by path length and by
relationship. The former, which we call \texttt{LEMMAPATH}, is similar to a window size
but counts the steps in a dependency path instead of slots in a bag-of-words window.
The latter, \texttt{LEMMAREL}, matches the dependency paths to specific templates inspired
by the context words tagged as informative by the annotators.

To exemplify, let's look at (1) and take \emph{herhalen} `to repeat' as the target.

\begin{enumerate}
\def\labelenumi{(\arabic{enumi})}
\tightlist
\item
  \emph{De geschiedenis rond Remmelink herhaalt zich.} `The history around Remmelink repeats itself.'
\end{enumerate}

\begin{description}
\tightlist
\item[\texttt{LEMMAPATH}]
This set of dependency-based models selects the features that enter a syntactic
relation with the target with a maximum number of steps.

The possible values we have included are \texttt{selection2} and \texttt{selection3}, which filter
out context words more than two or three steps away, respectively, and \texttt{weight}, which
gives a larger weight to context words that are closer in the dependency path.

A one-step dependency path is either the head of the target or its direct dependent.
Such features are included by both \texttt{selection2} and \texttt{selection3} and receive a weight of 1 in \texttt{weight}.
In (1) this includes the subject, \emph{geschiedenis} `history', and the reflexive pronoun \emph{zich},
which depend directly on it. If the target was \emph{geschiedenis} `history', \emph{herhalen} `to repeat',
its head, would be selected.

A two-step dependency path is either the head of the head of the target, the dependent of its dependent,
or its sibling. Such features are included by both \texttt{selection2} and \texttt{selection3} and receive a weight of 2/3 in \texttt{weight}.
In (1) this includes the determiner \emph{de} and the modifier \emph{rond} `around' directly depending on
a \emph{geschiedenis} `history'.

A three-step dependency path is either the head of the head of the head of the target,
the sibiling of the head of its head, the dependent of the dependent of its dependent,
or the dependent of a sibling. A typical case of the last path is the subject of a passive construction with a modal,
where the target is the verb in participium (\emph{belastingen} `taxes' in \emph{de belastingen moeten geheven worden} `the taxes must be levied').
Such features are included in \texttt{selection3} but excluded from \texttt{selection2} and receive a weight of 1/3 in \texttt{weight}.
In (1) this corresponds to \emph{Remmelink}, the object of \emph{rond} `around'.

Features more than 3 steps away from the target are always excluded.
While some features four steps away can be interesting, such as passive subjects of a verb with two modals, they are not that frequent and may not be worth the noise included by accepting all features with so many steps between them and the target. To catch those relationships, \texttt{LEMMAREL} is a more efficient method.
There are no context words more than three steps away from the target in (1).
\item[\texttt{LEMMAREL}]
This set of dependency-based models selects the features that enter in a certain
syntactic relation with the target. They are tailored to the part-of-speech of the target,
and each group expands on the selection of the group before it. The specific selections
are listed in Table \ref{tab:lemmareltable}.
\end{description}

\providecommand{\docline}[3]{\noalign{\global\setlength{\arrayrulewidth}{#1}}\arrayrulecolor[HTML]{#2}\cline{#3}}

\setlength{\tabcolsep}{2pt}

\renewcommand*{\arraystretch}{1.5}

\begin{longtable}[c]{|p{0.75in}|p{16.02in}|p{11.21in}|p{15.43in}}

\caption{Dependency paths selected by different `LEMMAREL` values.}\label{tab:lemmareltable}\\

\hhline{>{\arrayrulecolor[HTML]{666666}\global\arrayrulewidth=2pt}->{\arrayrulecolor[HTML]{666666}\global\arrayrulewidth=2pt}->{\arrayrulecolor[HTML]{666666}\global\arrayrulewidth=2pt}->{\arrayrulecolor[HTML]{666666}\global\arrayrulewidth=2pt}-}

\multicolumn{1}{!{\color[HTML]{000000}\vrule width 0pt}>{\raggedright}p{\dimexpr 0.75in+0\tabcolsep+0\arrayrulewidth}}{\fontsize{11}{11}\selectfont{\textcolor[HTML]{000000}{\global\setmainfont{Arial}groups}}} & \multicolumn{1}{!{\color[HTML]{000000}\vrule width 0pt}>{\raggedright}p{\dimexpr 16.02in+0\tabcolsep+0\arrayrulewidth}}{\fontsize{11}{11}\selectfont{\textcolor[HTML]{000000}{\global\setmainfont{Arial}nouns}}} & \multicolumn{1}{!{\color[HTML]{000000}\vrule width 0pt}>{\raggedright}p{\dimexpr 11.21in+0\tabcolsep+0\arrayrulewidth}}{\fontsize{11}{11}\selectfont{\textcolor[HTML]{000000}{\global\setmainfont{Arial}verbs}}} & \multicolumn{1}{!{\color[HTML]{000000}\vrule width 0pt}>{\raggedright}p{\dimexpr 15.43in+0\tabcolsep+0\arrayrulewidth}!{\color[HTML]{000000}\vrule width 0pt}}{\fontsize{11}{11}\selectfont{\textcolor[HTML]{000000}{\global\setmainfont{Arial}adjectives}}} \\

\noalign{\global\setlength{\arrayrulewidth}{2pt}}\arrayrulecolor[HTML]{666666}\cline{1-4}

\endfirsthead

\hhline{>{\arrayrulecolor[HTML]{666666}\global\arrayrulewidth=2pt}->{\arrayrulecolor[HTML]{666666}\global\arrayrulewidth=2pt}->{\arrayrulecolor[HTML]{666666}\global\arrayrulewidth=2pt}->{\arrayrulecolor[HTML]{666666}\global\arrayrulewidth=2pt}-}

\multicolumn{1}{!{\color[HTML]{000000}\vrule width 0pt}>{\raggedright}p{\dimexpr 0.75in+0\tabcolsep+0\arrayrulewidth}}{\fontsize{11}{11}\selectfont{\textcolor[HTML]{000000}{\global\setmainfont{Arial}groups}}} & \multicolumn{1}{!{\color[HTML]{000000}\vrule width 0pt}>{\raggedright}p{\dimexpr 16.02in+0\tabcolsep+0\arrayrulewidth}}{\fontsize{11}{11}\selectfont{\textcolor[HTML]{000000}{\global\setmainfont{Arial}nouns}}} & \multicolumn{1}{!{\color[HTML]{000000}\vrule width 0pt}>{\raggedright}p{\dimexpr 11.21in+0\tabcolsep+0\arrayrulewidth}}{\fontsize{11}{11}\selectfont{\textcolor[HTML]{000000}{\global\setmainfont{Arial}verbs}}} & \multicolumn{1}{!{\color[HTML]{000000}\vrule width 0pt}>{\raggedright}p{\dimexpr 15.43in+0\tabcolsep+0\arrayrulewidth}!{\color[HTML]{000000}\vrule width 0pt}}{\fontsize{11}{11}\selectfont{\textcolor[HTML]{000000}{\global\setmainfont{Arial}adjectives}}} \\

\noalign{\global\setlength{\arrayrulewidth}{2pt}}\arrayrulecolor[HTML]{666666}\cline{1-4}\endhead



\multicolumn{1}{!{\color[HTML]{000000}\vrule width 0pt}>{\raggedright}p{\dimexpr 0.75in+0\tabcolsep+0\arrayrulewidth}}{\fontsize{11}{11}\selectfont{\textcolor[HTML]{000000}{\global\setmainfont{Arial}1}}} & \multicolumn{1}{!{\color[HTML]{000000}\vrule width 0pt}>{\raggedright}p{\dimexpr 16.02in+0\tabcolsep+0\arrayrulewidth}}{\fontsize{11}{11}\selectfont{\textcolor[HTML]{000000}{\global\setmainfont{Arial}modifiers and determiners of the target, items of which the target is modifier or determiner, and verbs of which the target is object or subject}}} & \multicolumn{1}{!{\color[HTML]{000000}\vrule width 0pt}>{\raggedright}p{\dimexpr 11.21in+0\tabcolsep+0\arrayrulewidth}}{\fontsize{11}{11}\selectfont{\textcolor[HTML]{000000}{\global\setmainfont{Arial}direct objects, active and passive subjects (with up to two modals for the active one), reflexive complement and prepositions depending directly on the target}}} & \multicolumn{1}{!{\color[HTML]{000000}\vrule width 0pt}>{\raggedright}p{\dimexpr 15.43in+0\tabcolsep+0\arrayrulewidth}!{\color[HTML]{000000}\vrule width 0pt}}{\fontsize{11}{11}\selectfont{\textcolor[HTML]{000000}{\global\setmainfont{Arial}nouns modified by the target and direct modifiers of it (except for prepositions), subject and direct objects of the verbs of which the target is direct modifier or predicate complement, with up to one modal or auxiliary in between}}} \\





\multicolumn{1}{!{\color[HTML]{000000}\vrule width 0pt}>{\raggedright}p{\dimexpr 0.75in+0\tabcolsep+0\arrayrulewidth}}{\fontsize{11}{11}\selectfont{\textcolor[HTML]{000000}{\global\setmainfont{Arial}2}}} & \multicolumn{1}{!{\color[HTML]{000000}\vrule width 0pt}>{\raggedright}p{\dimexpr 16.02in+0\tabcolsep+0\arrayrulewidth}}{\fontsize{11}{11}\selectfont{\textcolor[HTML]{000000}{\global\setmainfont{Arial}conjuncts of the target (with or without conjunction), objects of the modifier of the target, and items of whose modifier the target is object}}} & \multicolumn{1}{!{\color[HTML]{000000}\vrule width 0pt}>{\raggedright}p{\dimexpr 11.21in+0\tabcolsep+0\arrayrulewidth}}{\fontsize{11}{11}\selectfont{\textcolor[HTML]{000000}{\global\setmainfont{Arial}conjuncts of the target, complementizers, nouns depending through a preposition and verbal complements or elements of which the target is a verbal complement}}} & \multicolumn{1}{!{\color[HTML]{000000}\vrule width 0pt}>{\raggedright}p{\dimexpr 15.43in+0\tabcolsep+0\arrayrulewidth}!{\color[HTML]{000000}\vrule width 0pt}}{\fontsize{11}{11}\selectfont{\textcolor[HTML]{000000}{\global\setmainfont{Arial}object of the preposition modifying the target, conjunct of the target (with or without conjunction), prepositional object of verb modified by target (as modifier or prepositional complement)}}} \\





\multicolumn{1}{!{\color[HTML]{000000}\vrule width 0pt}>{\raggedright}p{\dimexpr 0.75in+0\tabcolsep+0\arrayrulewidth}}{\fontsize{11}{11}\selectfont{\textcolor[HTML]{000000}{\global\setmainfont{Arial}3}}} & \multicolumn{1}{!{\color[HTML]{000000}\vrule width 0pt}>{\raggedright}p{\dimexpr 16.02in+0\tabcolsep+0\arrayrulewidth}}{\fontsize{11}{11}\selectfont{\textcolor[HTML]{000000}{\global\setmainfont{Arial}objects and modifiers of items of which the target is subject or modifier, subjects and modifiers of items of which the target is subject or modifier, modifiers of the modifiers of the target, and items of whose modifier the target is modifier}}} & \multicolumn{1}{!{\color[HTML]{000000}\vrule width 0pt}>{\raggedright}p{\dimexpr 11.21in+0\tabcolsep+0\arrayrulewidth}}{\fontsize{11}{11}\selectfont{\textcolor[HTML]{000000}{\global\setmainfont{Arial}}}} & \multicolumn{1}{!{\color[HTML]{000000}\vrule width 0pt}>{\raggedright}p{\dimexpr 15.43in+0\tabcolsep+0\arrayrulewidth}!{\color[HTML]{000000}\vrule width 0pt}}{\fontsize{11}{11}\selectfont{\textcolor[HTML]{000000}{\global\setmainfont{Arial}}}} \\

\noalign{\global\setlength{\arrayrulewidth}{2pt}}\arrayrulecolor[HTML]{666666}\cline{1-4}

\end{longtable}

\hypertarget{ppmi-weighting}{%
\subsection{PPMI weighting}\label{ppmi-weighting}}

The \texttt{PPMI} parameter is taken outside the set of first-order parameters because it can both filter out first-order features and reshape their vector representations. In truth, the choice of \textbf{p}ositive \textbf{p}ointwise \textbf{m}utual \textbf{i}nformation (PPMI) over other weighting mechanisms, as well as setting a threshold or not, is already a parameter setting, which in these circumstances is set to PPMI and a threshold of 0. In all cases, the PPMI was calculated based on a 4-4 window (that could also be a variable parameter).

This parameter can take three values. \texttt{selection} and \texttt{weight} mean that only the first-order features with a PPMI \textgreater{} 0 with the target type are selected, and the rest discarded, while \texttt{no} does not apply the filter. The difference between \texttt{selection} and \texttt{weight} is that the former only uses the value to filter the context features, while the latter also weighs their vectors with that value.

\hypertarget{second-order-selection}{%
\subsection{Second-order selection}\label{second-order-selection}}

The selection of second-order features influences the shape of the vectors: how the selected first-order features are represented. While the frequency transformation and the window on which such values were computed could be varied, they were set to fixed values, namely PPMI and 4-4 respectively. The parameters that were varied across, although we don't expect drastic differences between the models, are vector length and part-of-speech.

\begin{description}
\tightlist
\item[\texttt{SOC-POS} (second order part-of-speech)]
This parameter can take two values: \texttt{nav} and \texttt{all}. In the former case, a selection of 13771 lect-neutral nouns, adjectives and verbs made by Stefano is taken as the set of possible second-order features. In the latter, all lemmas with frequency above 227 and any part-of-speech are considered.
\item[\texttt{LENGTH}]
Vector length is the number of second-order features and therefore the dimensionality of the matrices on which the distance matrices are based, although the amount is not all that changes. It is applied after filtering by part-of-speech.

We have selected two values: \texttt{5000} and \texttt{FOC}. The former includes the 5000 most frequent elements of the possible features, while the latter takes the intersection between the possible second-order-features and the first-order-features, regardless of frequency. With \texttt{SOC-POS:all}, \texttt{FOC} will include all first-order features of that model, while with \texttt{SOC-POS:nav}, only those included in Stefano's selection.

The actual number of dimensions resulting from \texttt{FOC} depends on the strictness of the first order filter. This information can be found on the plots that, for each staal, show how many first order context words are left after each combination of first order filters.
\end{description}

\hypertarget{foc-as-soc}{%
\subsubsection{FOC as SOC}\label{foc-as-soc}}

What does it mean to use the same first-order context words as second-order context words?

First, depending on the number of target tokens and the strictness of the filter, there could be a different number of context words, ranging in the hundreds or low thousands.

Second, the context words will be compared based on their co-occurrence with each other. The behaviour of a context word outside the context of the target will be largely ignored: of course, the association strength between two items has to do with their co-occurrence across the whole corpus, as well as their non-co-occurrence, but it will only be included in the second order vector of the first item if the second is also among the first order context words.

\hypertarget{medoids}{%
\section{Medoids}\label{medoids}}

The multiple parameters return a huge number of models, and while purely quantitative methods might
be able to process and compare them, it is not feasible for a human to look at hundreds of clouds
and stay sane enough to make out anything from them. A more efficient --and easier on the human mind--
way to approach this is, instead, to look at representative models.

This method requires us to choose a number of medoids beforehand, which is not an easy
task. If we wanted the medoids to represent the best clustering solution, we could run
the algorithm with different values of \(k\) and compare the results with measures such
as silhouette width, as suggested by \textcite{levshina_2015}. However, that is not
necessarily our goal. We want to be able to see as much variation as possible, while keeping
the number of different models manageable (i.e.~below 9). It is not particularly problematic
if these models are redundant, as long as we can ensure that all the phenomena that we
are interested in are represented in them.

For example, given a lemma with multiple senses, it might be the case that some models
group the tokens of one sense, and others group the tokens of another: we would like to see
representatives of both kinds.

There is no guarantee that the method with the best silhouette returns all the variation we
are interested in --our goal is, rather, to limit the number of different models we need to
examine from the total number, say 200, to a more manageable amount, like 8.
In the same terms, there is also no guarantee that when we identify something interesting
in a medoid, i.e.~an island for a particular usage pattern, all the models in the cluster of
that medoid, and only those models, will share that characteristic. In order to check that,
we can look at random samples (again, of 8 or 9 models) of each of the clusters and
visually compare them to their medoids. This doesn't need to be as thorough an examination as
that of the medoids themselves: it suffices to check if the random sample is not too different
and seems to share the characteristic of interest. {[}add example{]}

In general terms, for the characteristics identified in the case studies that make up this
investigation, we can be quite confident that the medoids are representative of the models in
their clusters. However, depending on the concreteness of the phenomena, the variation across
models, the clarity of the visualization and the wishful thinking that might lurk in the
researchers' minds, it might be the case that something found or assessed in a medoid is not
shared by the models in its cluster. The comparison needed with the random sample should be
fast and honest and is strongly recommended: if the medoids are representative, you can see it
in an instant; if they are not, it just takes a bit longer to admit it. It is \emph{not} the same as actually studying and comparing 64 different models.

\hypertarget{from-corpora-to-clouds}{%
\chapter{From corpora to clouds}\label{from-corpora-to-clouds}}

The main goal of the distributional models discussed in this text is to explore semasiological structure
from textual data. The starting point is a corpus, and one of the most tangible outputs is the visual representation
as a cloud.

In this chapter we will describe the path needed to take to generate clouds from the raw,
seemingly indomitable ocean that is a corpus.

First, we will describe token-level vector space models are created.
Section \ref{vector-creation} will explain count-based models, but this is by no means the only viable path.
Other techniques, such as BERT \autocite{BERT}, that can generate vectors for individual instances of a word,
can be used for the first stage of this workflow.

Once we have token-level vectors, we need to process them. For visualization purposes,
we need to reduce the numerous dimensions of the vectors to a manageable number, such as 2.
Section \ref{dim-reduction} will explore and compare a few alternatives.

The same output that is put through dimensionality reduction for the visualization can also
be submitted to other forms of analysis such as clustering algorithms, whose results may
even be combined with the visualization. We will look into HDBSCAN in

\hypertarget{vector-creation}{%
\section{A cloud machine}\label{vector-creation}}

At the core of vector space models, \emph{aka} distributional models, we find the Distributional Hypothesis, which is most often linked to Harris's observation that ``difference of meaning correlates with difference of distribution'' \autocite*[ 156]{harris_1954}. In other words, items that occur in similar contexts in a given corpus will be semantically similar, while those that occur in different contexts will be semantically different. Crucially, this does not imply that we can describe an individual item with their distributional properties, but that comparing the distribution of two items can tell us something about their semantic relationship \autocite[ 19]{sahlgren_2006}.

Distributional models operationalize this idea by representing words as vectors (i.e.~arrays of numbers) coding frequency information. Typically, the raw frequency is transformed to some association strength measure, such as pointwise mutual information \autocite[PMI, see][]{church.hanks_1989}, which compares the frequency with which two words occur close to each other and the expected frequency if the words were independent. For example, Table \ref{tab:vec1} shows small vectors representing the English nouns \emph{linguistic}, \emph{lexicography}, \emph{research} and \emph{chocolate}, as well as the adjective \emph{computational}, as series of association strengths with a set of lemmas. Empty cells indicate that the word in the row and the word in the column never co-occur in the corpus (given a certain window span).

\providecommand{\docline}[3]{\noalign{\global\setlength{\arrayrulewidth}{#1}}\arrayrulecolor[HTML]{#2}\cline{#3}}

\setlength{\tabcolsep}{2pt}

\renewcommand*{\arraystretch}{1.5}

\begin{longtable}[c]{|p{0.75in}|p{0.75in}|p{0.75in}|p{0.75in}|p{0.75in}|p{0.75in}}

\caption{Example of type-level vectors.}\label{tab:vec1}\\

\hhline{>{\arrayrulecolor[HTML]{666666}\global\arrayrulewidth=2pt}->{\arrayrulecolor[HTML]{666666}\global\arrayrulewidth=2pt}->{\arrayrulecolor[HTML]{666666}\global\arrayrulewidth=2pt}->{\arrayrulecolor[HTML]{666666}\global\arrayrulewidth=2pt}->{\arrayrulecolor[HTML]{666666}\global\arrayrulewidth=2pt}->{\arrayrulecolor[HTML]{666666}\global\arrayrulewidth=2pt}-}

\multicolumn{1}{!{\color[HTML]{000000}\vrule width 0pt}>{\raggedright}p{\dimexpr 0.75in+0\tabcolsep+0\arrayrulewidth}}{\fontsize{11}{11}\selectfont{\textcolor[HTML]{000000}{\global\setmainfont{Arial}target}}} & \multicolumn{1}{!{\color[HTML]{000000}\vrule width 0pt}>{\raggedleft}p{\dimexpr 0.75in+0\tabcolsep+0\arrayrulewidth}}{\fontsize{11}{11}\selectfont{\textcolor[HTML]{000000}{\global\setmainfont{Arial}language/n}}} & \multicolumn{1}{!{\color[HTML]{000000}\vrule width 0pt}>{\raggedleft}p{\dimexpr 0.75in+0\tabcolsep+0\arrayrulewidth}}{\fontsize{11}{11}\selectfont{\textcolor[HTML]{000000}{\global\setmainfont{Arial}word/n}}} & \multicolumn{1}{!{\color[HTML]{000000}\vrule width 0pt}>{\raggedleft}p{\dimexpr 0.75in+0\tabcolsep+0\arrayrulewidth}}{\fontsize{11}{11}\selectfont{\textcolor[HTML]{000000}{\global\setmainfont{Arial}flemish/j}}} & \multicolumn{1}{!{\color[HTML]{000000}\vrule width 0pt}>{\raggedleft}p{\dimexpr 0.75in+0\tabcolsep+0\arrayrulewidth}}{\fontsize{11}{11}\selectfont{\textcolor[HTML]{000000}{\global\setmainfont{Arial}english/j}}} & \multicolumn{1}{!{\color[HTML]{000000}\vrule width 0pt}>{\raggedleft}p{\dimexpr 0.75in+0\tabcolsep+0\arrayrulewidth}!{\color[HTML]{000000}\vrule width 0pt}}{\fontsize{11}{11}\selectfont{\textcolor[HTML]{000000}{\global\setmainfont{Arial}speak/v}}} \\

\noalign{\global\setlength{\arrayrulewidth}{2pt}}\arrayrulecolor[HTML]{666666}\cline{1-6}

\endfirsthead

\hhline{>{\arrayrulecolor[HTML]{666666}\global\arrayrulewidth=2pt}->{\arrayrulecolor[HTML]{666666}\global\arrayrulewidth=2pt}->{\arrayrulecolor[HTML]{666666}\global\arrayrulewidth=2pt}->{\arrayrulecolor[HTML]{666666}\global\arrayrulewidth=2pt}->{\arrayrulecolor[HTML]{666666}\global\arrayrulewidth=2pt}->{\arrayrulecolor[HTML]{666666}\global\arrayrulewidth=2pt}-}

\multicolumn{1}{!{\color[HTML]{000000}\vrule width 0pt}>{\raggedright}p{\dimexpr 0.75in+0\tabcolsep+0\arrayrulewidth}}{\fontsize{11}{11}\selectfont{\textcolor[HTML]{000000}{\global\setmainfont{Arial}target}}} & \multicolumn{1}{!{\color[HTML]{000000}\vrule width 0pt}>{\raggedleft}p{\dimexpr 0.75in+0\tabcolsep+0\arrayrulewidth}}{\fontsize{11}{11}\selectfont{\textcolor[HTML]{000000}{\global\setmainfont{Arial}language/n}}} & \multicolumn{1}{!{\color[HTML]{000000}\vrule width 0pt}>{\raggedleft}p{\dimexpr 0.75in+0\tabcolsep+0\arrayrulewidth}}{\fontsize{11}{11}\selectfont{\textcolor[HTML]{000000}{\global\setmainfont{Arial}word/n}}} & \multicolumn{1}{!{\color[HTML]{000000}\vrule width 0pt}>{\raggedleft}p{\dimexpr 0.75in+0\tabcolsep+0\arrayrulewidth}}{\fontsize{11}{11}\selectfont{\textcolor[HTML]{000000}{\global\setmainfont{Arial}flemish/j}}} & \multicolumn{1}{!{\color[HTML]{000000}\vrule width 0pt}>{\raggedleft}p{\dimexpr 0.75in+0\tabcolsep+0\arrayrulewidth}}{\fontsize{11}{11}\selectfont{\textcolor[HTML]{000000}{\global\setmainfont{Arial}english/j}}} & \multicolumn{1}{!{\color[HTML]{000000}\vrule width 0pt}>{\raggedleft}p{\dimexpr 0.75in+0\tabcolsep+0\arrayrulewidth}!{\color[HTML]{000000}\vrule width 0pt}}{\fontsize{11}{11}\selectfont{\textcolor[HTML]{000000}{\global\setmainfont{Arial}speak/v}}} \\

\noalign{\global\setlength{\arrayrulewidth}{2pt}}\arrayrulecolor[HTML]{666666}\cline{1-6}\endhead



\multicolumn{6}{!{\color[HTML]{FFFFFF}\vrule width 0pt}>{\raggedright}p{\dimexpr 4.5in+10\tabcolsep+5\arrayrulewidth}}{\fontsize{11}{11}\selectfont{\textcolor[HTML]{000000}{\global\setmainfont{Arial}PMI values based on symmetric window of 10; frequency data from GloWbE.}}} \\

\endfoot



\multicolumn{1}{!{\color[HTML]{000000}\vrule width 0pt}>{\raggedright}p{\dimexpr 0.75in+0\tabcolsep+0\arrayrulewidth}}{\fontsize{11}{11}\selectfont{\textcolor[HTML]{000000}{\global\setmainfont{Arial}linguistics/n}}} & \multicolumn{1}{!{\color[HTML]{000000}\vrule width 0pt}>{\raggedleft}p{\dimexpr 0.75in+0\tabcolsep+0\arrayrulewidth}}{\fontsize{11}{11}\selectfont{\textcolor[HTML]{000000}{\global\setmainfont{Arial}4.37}}} & \multicolumn{1}{!{\color[HTML]{000000}\vrule width 0pt}>{\raggedleft}p{\dimexpr 0.75in+0\tabcolsep+0\arrayrulewidth}}{\fontsize{11}{11}\selectfont{\textcolor[HTML]{000000}{\global\setmainfont{Arial}0.99}}} & \multicolumn{1}{!{\color[HTML]{000000}\vrule width 0pt}>{\raggedleft}p{\dimexpr 0.75in+0\tabcolsep+0\arrayrulewidth}}{\fontsize{11}{11}\selectfont{\textcolor[HTML]{000000}{\global\setmainfont{Arial}}}} & \multicolumn{1}{!{\color[HTML]{000000}\vrule width 0pt}>{\raggedleft}p{\dimexpr 0.75in+0\tabcolsep+0\arrayrulewidth}}{\fontsize{11}{11}\selectfont{\textcolor[HTML]{000000}{\global\setmainfont{Arial}3.16}}} & \multicolumn{1}{!{\color[HTML]{000000}\vrule width 0pt}>{\raggedleft}p{\dimexpr 0.75in+0\tabcolsep+0\arrayrulewidth}!{\color[HTML]{000000}\vrule width 0pt}}{\fontsize{11}{11}\selectfont{\textcolor[HTML]{000000}{\global\setmainfont{Arial}0.41}}} \\





\multicolumn{1}{!{\color[HTML]{000000}\vrule width 0pt}>{\raggedright}p{\dimexpr 0.75in+0\tabcolsep+0\arrayrulewidth}}{\fontsize{11}{11}\selectfont{\textcolor[HTML]{000000}{\global\setmainfont{Arial}lexicography/n}}} & \multicolumn{1}{!{\color[HTML]{000000}\vrule width 0pt}>{\raggedleft}p{\dimexpr 0.75in+0\tabcolsep+0\arrayrulewidth}}{\fontsize{11}{11}\selectfont{\textcolor[HTML]{000000}{\global\setmainfont{Arial}3.51}}} & \multicolumn{1}{!{\color[HTML]{000000}\vrule width 0pt}>{\raggedleft}p{\dimexpr 0.75in+0\tabcolsep+0\arrayrulewidth}}{\fontsize{11}{11}\selectfont{\textcolor[HTML]{000000}{\global\setmainfont{Arial}2.18}}} & \multicolumn{1}{!{\color[HTML]{000000}\vrule width 0pt}>{\raggedleft}p{\dimexpr 0.75in+0\tabcolsep+0\arrayrulewidth}}{\fontsize{11}{11}\selectfont{\textcolor[HTML]{000000}{\global\setmainfont{Arial}}}} & \multicolumn{1}{!{\color[HTML]{000000}\vrule width 0pt}>{\raggedleft}p{\dimexpr 0.75in+0\tabcolsep+0\arrayrulewidth}}{\fontsize{11}{11}\selectfont{\textcolor[HTML]{000000}{\global\setmainfont{Arial}2.19}}} & \multicolumn{1}{!{\color[HTML]{000000}\vrule width 0pt}>{\raggedleft}p{\dimexpr 0.75in+0\tabcolsep+0\arrayrulewidth}!{\color[HTML]{000000}\vrule width 0pt}}{\fontsize{11}{11}\selectfont{\textcolor[HTML]{000000}{\global\setmainfont{Arial}2.09}}} \\





\multicolumn{1}{!{\color[HTML]{000000}\vrule width 0pt}>{\raggedright}p{\dimexpr 0.75in+0\tabcolsep+0\arrayrulewidth}}{\fontsize{11}{11}\selectfont{\textcolor[HTML]{000000}{\global\setmainfont{Arial}computational/j}}} & \multicolumn{1}{!{\color[HTML]{000000}\vrule width 0pt}>{\raggedleft}p{\dimexpr 0.75in+0\tabcolsep+0\arrayrulewidth}}{\fontsize{11}{11}\selectfont{\textcolor[HTML]{000000}{\global\setmainfont{Arial}1.60}}} & \multicolumn{1}{!{\color[HTML]{000000}\vrule width 0pt}>{\raggedleft}p{\dimexpr 0.75in+0\tabcolsep+0\arrayrulewidth}}{\fontsize{11}{11}\selectfont{\textcolor[HTML]{000000}{\global\setmainfont{Arial}0.08}}} & \multicolumn{1}{!{\color[HTML]{000000}\vrule width 0pt}>{\raggedleft}p{\dimexpr 0.75in+0\tabcolsep+0\arrayrulewidth}}{\fontsize{11}{11}\selectfont{\textcolor[HTML]{000000}{\global\setmainfont{Arial}}}} & \multicolumn{1}{!{\color[HTML]{000000}\vrule width 0pt}>{\raggedleft}p{\dimexpr 0.75in+0\tabcolsep+0\arrayrulewidth}}{\fontsize{11}{11}\selectfont{\textcolor[HTML]{000000}{\global\setmainfont{Arial}-1.00}}} & \multicolumn{1}{!{\color[HTML]{000000}\vrule width 0pt}>{\raggedleft}p{\dimexpr 0.75in+0\tabcolsep+0\arrayrulewidth}!{\color[HTML]{000000}\vrule width 0pt}}{\fontsize{11}{11}\selectfont{\textcolor[HTML]{000000}{\global\setmainfont{Arial}-1.80}}} \\





\multicolumn{1}{!{\color[HTML]{000000}\vrule width 0pt}>{\raggedright}p{\dimexpr 0.75in+0\tabcolsep+0\arrayrulewidth}}{\fontsize{11}{11}\selectfont{\textcolor[HTML]{000000}{\global\setmainfont{Arial}research/n}}} & \multicolumn{1}{!{\color[HTML]{000000}\vrule width 0pt}>{\raggedleft}p{\dimexpr 0.75in+0\tabcolsep+0\arrayrulewidth}}{\fontsize{11}{11}\selectfont{\textcolor[HTML]{000000}{\global\setmainfont{Arial}0.20}}} & \multicolumn{1}{!{\color[HTML]{000000}\vrule width 0pt}>{\raggedleft}p{\dimexpr 0.75in+0\tabcolsep+0\arrayrulewidth}}{\fontsize{11}{11}\selectfont{\textcolor[HTML]{000000}{\global\setmainfont{Arial}-0.84}}} & \multicolumn{1}{!{\color[HTML]{000000}\vrule width 0pt}>{\raggedleft}p{\dimexpr 0.75in+0\tabcolsep+0\arrayrulewidth}}{\fontsize{11}{11}\selectfont{\textcolor[HTML]{000000}{\global\setmainfont{Arial}0.04}}} & \multicolumn{1}{!{\color[HTML]{000000}\vrule width 0pt}>{\raggedleft}p{\dimexpr 0.75in+0\tabcolsep+0\arrayrulewidth}}{\fontsize{11}{11}\selectfont{\textcolor[HTML]{000000}{\global\setmainfont{Arial}-0.50}}} & \multicolumn{1}{!{\color[HTML]{000000}\vrule width 0pt}>{\raggedleft}p{\dimexpr 0.75in+0\tabcolsep+0\arrayrulewidth}!{\color[HTML]{000000}\vrule width 0pt}}{\fontsize{11}{11}\selectfont{\textcolor[HTML]{000000}{\global\setmainfont{Arial}-0.38}}} \\





\multicolumn{1}{!{\color[HTML]{000000}\vrule width 0pt}>{\raggedright}p{\dimexpr 0.75in+0\tabcolsep+0\arrayrulewidth}}{\fontsize{11}{11}\selectfont{\textcolor[HTML]{000000}{\global\setmainfont{Arial}chocolate/n}}} & \multicolumn{1}{!{\color[HTML]{000000}\vrule width 0pt}>{\raggedleft}p{\dimexpr 0.75in+0\tabcolsep+0\arrayrulewidth}}{\fontsize{11}{11}\selectfont{\textcolor[HTML]{000000}{\global\setmainfont{Arial}-1.72}}} & \multicolumn{1}{!{\color[HTML]{000000}\vrule width 0pt}>{\raggedleft}p{\dimexpr 0.75in+0\tabcolsep+0\arrayrulewidth}}{\fontsize{11}{11}\selectfont{\textcolor[HTML]{000000}{\global\setmainfont{Arial}-0.53}}} & \multicolumn{1}{!{\color[HTML]{000000}\vrule width 0pt}>{\raggedleft}p{\dimexpr 0.75in+0\tabcolsep+0\arrayrulewidth}}{\fontsize{11}{11}\selectfont{\textcolor[HTML]{000000}{\global\setmainfont{Arial}1.28}}} & \multicolumn{1}{!{\color[HTML]{000000}\vrule width 0pt}>{\raggedleft}p{\dimexpr 0.75in+0\tabcolsep+0\arrayrulewidth}}{\fontsize{11}{11}\selectfont{\textcolor[HTML]{000000}{\global\setmainfont{Arial}-0.73}}} & \multicolumn{1}{!{\color[HTML]{000000}\vrule width 0pt}>{\raggedleft}p{\dimexpr 0.75in+0\tabcolsep+0\arrayrulewidth}!{\color[HTML]{000000}\vrule width 0pt}}{\fontsize{11}{11}\selectfont{\textcolor[HTML]{000000}{\global\setmainfont{Arial}-1.13}}} \\

\noalign{\global\setlength{\arrayrulewidth}{2pt}}\arrayrulecolor[HTML]{666666}\cline{1-6}

\end{longtable}

Each row is a vector coding the distributional information of the lemma it represents. As we can see in this example, words with similar vectors (e.g.~\emph{linguistics} and \emph{lexicography}) are semantically similar, while words with different vectors (e.g.~\emph{linguistics} and \emph{chocolate}) are semantically different.

The vectors in Table \ref{tab:vec1} are type-level vectors: each of them aggregates over all the instances of a given word, e.g.~\emph{linguistics}, to build an overall profile. As a result, it collapses the internal variation of the lemma, i.e.~its semasiological structure. In order to uncover such information, we need to build vectors for the individual instances or tokens, relying on the same principle: items occurring in similar contexts will be semantically similar. For instance, we might want to model the three (artificial) occurrences of \emph{study} in (2) through (4), where the target item is in italics.

\begin{enumerate}
\def\labelenumi{(\arabic{enumi})}
\setcounter{enumi}{1}
\tightlist
\item
  Would you like to \emph{study} lexicography?
\item
  They \emph{study} this in computational linguistics as well.
\item
  I eat chocolate while I \emph{study}.
\end{enumerate}

Given that, at the aggregate level, a word can co-occur with thousands of different words, type-level vectors can include thousands of values. In contrast, token-level vectors can only have as many values as the individual window size comprises, which drastically reduces the chances of overlap between vectors. In fact, the three examples don't share any item other than the target. As a solution, inspired by \textcite{schutze_1998}, we replace the context words around the token with their respective type-level vectors \autocite{heylen.etal_2015,depascale_2019}.

For example, we could represent example (2) with the vector for its context word \emph{lexicography}, that is, the second row in Table \ref{tab:vec1}; example (3) with the sum of the vectors for \emph{linguistics} (row 1) and \emph{computational} (row 3); and example (4) with the vector for \emph{chocolate} (row 5). This not only solves the sparsity issue, ensuring overlap between the vectors, but also allows us to find similarity between (2) and (3) based on the similarity between the vectors for \emph{lexicography} and \emph{linguistics}.

From applying this method we obtain numerical representations of occurrences of a word. We can compare them to each other by calculating pairwise distances, which is at the base of clustering analyses and visualization techniques based on dimensionality reduction. However, in order to obtain this result we need to make a number of decisions involved, mostly, in defining the context that will be used to represent an item \autocite[Cf.][ 83]{bolognesi_2020}.

\hypertarget{dim-reduction}{%
\section{Dimensionality reduction}\label{dim-reduction}}

Dimensionality reduction refers to algorithms that try to locate different items on a low-dimensional space (e.g.~2D) preserving their distances in the high-dimensional space (e.g.~5000D) as well as possible.
The literature up to today tends to go for either multidimensional scaling (MDS) or t-stochastic neighbor embeddings (t-SNE),
which may be run on R with the function \texttt{metaMDS()} of the \{vegan\} package \autocite{R-vegan}
and \texttt{Rtsne()} of the homonymous package \autocite{R-Rtsne}, respectively.

MDS is an ordination technique, like principal components analysis (PCA). It tries out different low-dimensional configurations and tries to maximize the correlation between the pairwise distances in the high-dimensional space and those in the low-dimensional space: items that are close together in one space should stay close together in the other, and items that are far apart in one space should stay far apart in the other.
It can be evaluated via the stress level, the complement of the correlation coefficient: if the correlation between the pairwise distances is 0.85, the stress level is 0.15.
Unlike PCA, however, the dimensions are not meaningful \emph{per se}; two different runs of MDS may result in plots that mirror each other while representing the same thing. Nonethelesss, the R implementation rotates the plot so that the horizontal axis represents the maximum variation.
In cognitive linguistics literature both metric \autocite{hilpert.correiasaavedra_2017,hilpert.flach_2020,koptjevskaja-tamm.sahlgren_2014}
and nonmetric MDS \autocite{heylen.etal_2015,heylen.etal_2012,depascale_2019,perek_2016} have been used.

The second technique, t-SNE \autocite{Rtsne2008,Rtsne2014}, has also been incorporated in cognitive distributional semantics \autocite{depascale_2019,perek_2018}.
The algorithm is quite different from MDS, but for our purposes the crucial point is that it prioritizes preserving local similarity structure instead of the global structure: items that are close together in the high-dimensional space should stay close together in the low-dimensional space, but those that are far apart in the high-dimensional space may be even farther apart in low-dimensional space. This leads to nice, tight clusters but the distance between them is less interpretable than in an MDS plot.
T-SNE was the state-of-the-art visualization technique for word vectors in computational linguistics \autocite{smilkov.etal_2016} but is now generally being replaced by UMAP.

In both cases we need to state the desired number of dimensions before running the algorithm --for visualization purposes, the most useful choice is 2. Three dimensions are difficult to interpret if projected on a 2D space, such as a screen \autocites[ 18]{card.etal_1999}[ 222]{wielfaert.etal_2019}. In addition, t-SNE requires setting a parameter called perplexity, which basically sets how many neighbors the preserved local structure should cover.

\hypertarget{nephovis}{%
\chapter{NephoVis}\label{nephovis}}

In this chapter we will learn how to use the visualization tool to explore
and compare token-level vector space models.

As of this moment, the tool can be found at a
\href{https://qlvl.github.io/NephoVis}{Github Page}, that is,
a \href{https://github.com/qlvl/NephoVis}{Github repository} that can be rendered
as a static website \autocite{montes.qlvl_2021}.
It obtains its data from a \href{https://github.com/qlvl/tokenclouds}{submodule};
an interested user could clone the repository and just modify the path to the data.

The code for the visualization is written in Javascript, making heavy use of the
\href{https://d3.js}{D3.js} library, which was designed for beautiful web-based
\textbf{d}ata-\textbf{d}riven visualization. While it is known of its steep learning curve,
it can be useful to think of it in terms of R's vectorized approach: it links
DOM elements to arrays and manipulates them based on the items' properties.

The main rationale and framework for this visualization tool was developed by
Thomas Wielfaert \autocite{wielfaert.etal_2019}; that code can be found
\href{https://github.com/tokenclouds/tokenclouds.github.io/LeTok/}{here}.

The current implementation would not exist without this foundational setup. However,
a number of the available features were added later.

The description of the tool will not immediately follow the expected workflow of an user.
Instead, we will start with the lowest level, \protect\hyperlink{level_3}{Level 3},
which represents individual token-level clouds,
zoom out into \protect\hyperlink{level_2}{Level 2}, which shows multiple token-level clouds simultaneously,
which will make the most abstract level, \protect\hyperlink{level_1}{Level 1}, more clear.
Afterwards (Section \ref{workflow}) we will briefly simulate the path an user would take from Level 1 to Level 3.
This perspective was also taken in \textcite{montes.heylen_Submitted}.
Finally, Section \ref{wishlist} goes into the Beta features
that require better development and testing, as well as ideas that we might want to
implement in the future. In any case, it must be noted that from July 2019 to
2021-05-01 the user and developer have been the same person, with occasional,
valuable input from other members of the Nephological Semantics project. The
project could certainly benefit from a wider input of suggestions.

\hypertarget{level_3}{%
\section{Level 3}\label{level_3}}

Level 3 of the visualization tool shows a zoomable scatterplot in which each glyph represents a token, i.e.~an instance of the target lexical item. The name of the model, coding the parameter settings as described in
, is indicated on the top. It is possible to map colors and shapes to categorical variables (such as sense labels) and sizes to numerical variables (such as number of available context words) and to select tokens with a given value by clicking on the corresponding legend key.

\hypertarget{level_2}{%
\section{Level 2}\label{level_2}}

Level 2 of the visualization is \emph{not} a scatterplot matrix, although it looks like one and the code was inspired by Mike Bostock's example. Instead, it is just an array of small plots next to each other and wrap of easier readability.

Each of them represents a different model and the same basic features from Level 3 are available: color, shape and size coding, selection by clicking and brushing, and finding the context by hovering over the tokens.

Because they are model-dependent, highlighted context and searching tokens by context word are meaningless in this level, where multiple models are being shown simultaneously. The key contribution of this level, next to the superficial visual comparison of the shape of each plot, is the ability to select one or more tokens in a plot and highlighting them in the rest of the plots as well. Thanks to this functionality, the user can compare the relative position of a group of tokens in a model against that in a different model.

\hypertarget{level_1}{%
\section{Level 1}\label{level_1}}

Level 1 shows one zoomable scatterplot, similar to Level 3, but with each glyph representing one model, instead of one token. As a reminder of the difference, the default shape in Level 1 is a wye (``Y''), while that in the other levels is a circle. The data represented by this scatterplot is not the distance between tokens anymore, but that between models, as described at the beginning of Section 3. This scatterplot aims to represent the similarity between models and allows the user to select the models to inspect according to different criteria. Categorical variables (e.g.~whether sentence boundaries are used) can be mapped to colors and shapes, as shown in Figure 5, and numerical variables (e.g.~number of tokens in the model) can be mapped to size. A selection of buttons on the left panel, as well as the legends for color and shape, can be used to filter models with a certain parameter setting. Otherwise, models can be selected by clicking on the glyphs that represent them.

\hypertarget{workflow}{%
\section{The full story}\label{workflow}}

The increasing granularity from Level 1 to Level 3 and the manner of access to different functionalities respect the mantra ``Overview first, zoom and filter, then details-on-demand'' \autocite[ 97]{shneiderman_1996}.
The individual plots in Levels 1 and 3 are literally zoomable; and in all cases it is possible to select items (either models, in Level 1, or tokens, in the other two), for more detailed inspection. Finally, a number of features show details on demand, such as the names of the models in Level 1 and the context of the tokens in the other two levels.

In practice, the user will start with Level 1, the scatterplot of models, and can look for structure in the distribution of the parameters on the plot. For example, color coding may reveal that models with nouns, adjectives, verbs and adverbs as first-order context words are very different from those without strong filters for part-of-speech, while the use of sentence boundaries makes little difference. Depending on whether the user wants to compare models similar or different to each other, or which parameters they would like to keep fixed, they will use individual selection or the buttons to choose models for Level 2. In our case, we click on ``Select medoids'', which selects the 8 models returned by a partitioning algorithm, which offers a wide range of variation in a manageable number of plots.

In Level 2 the user can already compare the shapes that the models take in their respective plots, the distribution of categories like sense labels, and the number of lost tokens. In addition, the ``distance matrix'' button offers a heatmap of the pairwise distances between the selected models. In the case of heffen, the restrictive collocational patterns it presents lead to crisp clusters in the visualization and consistent organization across models. However, models with less clearly defined structure may prove harder to understand. In both cases, the brushing and linking functionality highlights whether tokens that are grouped in one model are also grouped in a different model. From here, the user might switch back and forth between Level 2 and Level 3 for a more detailed inspection of the models.

\hypertarget{examining-context-words}{%
\subsection{Examining context words}\label{examining-context-words}}

While it is possible to look at the individual context of each token by hovering over them, it loses track of the larger patterns we want to understand. That is the purpose of the frequency tables in levels 2 and 3.

In any given model, tokens might be close together because they share a context word, and/or because their context words are (based on the second-order modelling) similar to each other. First-order parameters are, by definition, directly responsible for the selection of context words that will be used to model each token. Therefore, when inspecting a model, we might want to know which context word(s) pull certain tokens together, or why tokens that we expect to be together are far apart instead. In other words, if each model offers a different perspective on the distributional behavior of a token, we want to understand what informs said perspective.

In Level 3, individual tokens and groups of them may be selected in different ways. Given such a selection, clicking on ``Frequency table'' will open a table with one row per context word, a column indicating in how many of the selected tokens it occurs, and more columns with pre-computed information (e.g.~PMI values).

The following five columns include pre-computed frequency information, such as the raw co-occurrence frequency and PMI value between the context word and the target based on windows of 10 and 4, and raw frequency in the corpus.
These values can be interesting if we would like to strengthen or weaken filters for a smarter selection of context words. This particular model uses dependency-based information as well as a PMI threshold of 0 to select context words.

In Level 2, while comparing different models, the frequency table takes a different form. There is still one context word per row, but the number of tokens with which it co-occurs will depend on the model.
The columns in this table are all computed by the visualization based on the lists of context words per token per model. Next to the column with the name of the context word, the default table shows one column called ``total'' and one per model, headed by the corresponding number. The columns for each model match the second column in their Level 3 frequency table: they indicate with how many of the selected tokens the context word co-occurs. The ``total'' column, in contrast, reveals the union of this selection: with how many of the selected tokens the context word co-occurs in at least one model.

The default table counts how many of the selected tokens co-occur with each of the context words, but it does not use information from other tokens outside the selection, i.e.~the cue validity or association strength of the context words for the selected group. For that purpose, a dropdown button in the top left corner of the frequency table offers a small range of transformations, such as odds ratio, Fisher Exact, cue validity, etc. One such option shows the absolute frequencies within and outside the selection, where the green columns count the number of selected tokens that co-occur with each context word, and the white columns count the number of tokens outside of the selection co-occurring with those context words.

\hypertarget{wishlist}{%
\section{Wishlist}\label{wishlist}}

\hypertarget{hdbscan}{%
\chapter{HDBSCAN}\label{hdbscan}}

The visual exploration is extremely useful for a thorough, qualitative description
of the vector space models. However, such an application can also become an obstacle
to a truly systematic, scientific description. I would avoid talking about objectivity:
neither of us, individually, can truly be objective, and we should instead strive for a
humble admission of our own partiality and a fruitful combination of all our partialities.

When describing a cloud ---in particular these clouds that refuse to show clear images,
perfect sense disambiguation, distinct clusters---, how can we ensure that what we
see will be found by other researchers? How can we make our observations, if not
inherently valid, at least reproducible?
This is, after all, the goal we strive for when we embark on quantitative methods.

One tool that can help us systematize our observations, such as the tightness or at least
existEnce of distinct islands on a plot, is
Hierarchical Density-Based Spatial Clustering of Applications (HDBSCAN) \autocite{campello.etal_2013}.
This algorithm basically tries to distinguish dense areas separated by less dense areas\footnote{For a friendly description of how the algorithm works, the reader is directed to \textcite{mcinnes.etal_2016}; for an even friendlier explanation, find the authors on YouTube.}
and allows for noisy data. In other words, unlike traditional hierarchical clustering,
it will not try to cluster all of the points in the dataset, but instead may discard those
that are too far from everything else. Moreover, in comparison to its non hierarchical counterpart,
DBSCAN,
it requires only one parameter to be set \emph{a priori}, namely \(minPts\).

The \(minPts\) parameter indicates the minimum size of a dense group of points to be considered
a cluster. An isolated dense group of points for which \(n < minPts\) will be considered noise.
In the case studies described here we have fixed \(minPts\) to 8, which seems a reasonable
size for the smallest clusters, but it would be interested to look more systematically into
the effect of lowering this threshold. Rising the threshold, on the other hand, would
increase the proportion of points that are considered noise, which is already very high.

Like other clustering algorithms and the dimensionality reduction techniques,
HDBSCAN can take a token-context matrix as input or a distance matrix. We have used
the transformed distance matrix, that is, the same input fed into the t-SNE algorithm,
for the \texttt{hdbscan()} function of the \{dbscan\} R package \autocite{R-dbscan}. The output includes,
among other things, the cluster assignment, with noise points assigned to a cluster 0,
and epsilon values, which can be used as an estimate of density.

HDBSCAN estimates the density of {[}the area in which we find{]} a point \(a\) by calculating its
core distance \(core_{k}(a)\), which is the distance to its \(k\) nearest neighbor, \(k\) being \(minPts - 1\).
Then it recalculates the distance matrix by defining a new distance measure, called
\emph{mutual reachability}, which is defined as the maximum between the distance between
the items \(d(a, b)\) and each of their core distances.

\[d_{mreach-k}(a,b) = max(core_{k}(a), core_{k}(b), d(a,b))\]

Once the algorithm obtains these distances, it uses a single linkage method to create
hierarchical clusters, then using again \(minPts\) and other calculations to merge them
into the final selection. The \texttt{eps} (epsilon) values returned by \texttt{hdbscan()} indicate
the height, in the single linkeage tree, at which each point was joined to a cluster,
and can thus be used as a proxy for its ``density''.

This is intuitively more clear if we map the clustering solution to colors and the
\texttt{eps} value to transparency in a t-SNE plot with perplexity 30. For the most part,
the results converge, which has two main upsides. In the first place, we have independent
confirmation of the structure found by t-SNE, as a different algorithm processing
the same input returns compatible output. Second, insofar the HDBSCAN output matches
our visual assessment, it can systematize it and render it reproducible.

This wonders notwithstanding, the compatibility between the HDBSCAN output and visual examination is not guaranteed. We might find interesting tokens that are discarded as noise, or structure within
a single HDBSCAN cluster. However, it must be noted that this match (or lack thereof)
has been assessed only between \texttt{dbscan::hdbscan()} with \(minPts = 8\) and
\texttt{Rtsne::Rtsne()} with \(perplexity = 30\). It would be an interesting avenue for further research to experiment with other combinations, and of course UMAP output.

\hypertarget{annotation-schema}{%
\chapter{Annotation schema}\label{annotation-schema}}

For these case studies, we selected 34 Dutch lemmas to annotate and model with token level vector spaces, two of which (\emph{herkennen} and \emph{spoor}) were discarded.
The selection process will be presented in Section \ref{selection}, with a description of the selected items and what we expect their annotation to look like (eventually, what we expect the clouds to look like as well). A short description of the corpus \autocite[QLVLNewsCorpus,][]{depascale_2019} will follow. The annotation procedure will be the focus of the \ref{annotation}.
Students (later called \emph{annotators}) were recruited and hired to manually annotate samples of the selected lemmas, and while the administrative procedure itself is not of great interest to the project, a number of practical issues will be discussed: the distribution of tokens, the assignment of tasks (in particular, the graphic interface provided) and the processing/analysis of the data.

\hypertarget{selection}{%
\section{Selection of items}\label{selection}}

For this case study 34 Dutch lexical items were selected. We aimed to cover a variety of polysemy phenomena, which will be addressed in the specific sections for each part of speech: section \ref{nouns} for nouns, section \ref{adjs} for adjectives, and section \ref{verbs} for verbs.
By modelling different parts of speech and different kinds of polysemy, we expect to develop more robust generalizations regarding the parameter settings that best model specific phenomena.

The selection procedure mixed some introspection (thinking of words that could be interesting), looking up lexical resources (going through a tentative list of dictionary entries to figure out what kind of polysemy we could expect) and corpus data (surveying a sample of concordances for evidence of the expected polysemy). While dictionaries were an essential resource to sketch the sense labels the annotators would have to choose from, we also adjusted them to a more manageable granularity. The concordances were also crucial to estimate sense distribution and adjust the granularity of the definitions. We didn't want an overwhelmingly frequent sense to affect the annotators judgement, and very infrequent senses might be hard to model or at least to visualise in the 2D representations. Still, we did allow for some complexity and subtlety in some cases.

In a number of cases, the corpus survey (reading a concordance of 40-50 randomly selected instances) invalidated options that intuitively or according to the dictionary definitions would have conformed to our requirements. When judging such a discrepancy, it is important to take into account the composition of the corpus. The topics addressed in newspapers and the terms used to talk about them are not representative of everyday life or the entirety of language. Only as long as we keep these limitations in mind, can we draw valid conclusions from our data.

We also had cases of adjectives that could be used in adverbial form and weren't always properly tagged for part-of-speech, so we had to discard them. We made a difference between cases such as \emph{hoopvol} `hopeful', which often occurs in predicative contexts with a verb that is not copula (e.g.~\emph{ik ben hoopvol gestemd} `it makes me hopeful', \emph{hij kijkt hoopvol omhoog} `he looks up hopeful(ly)') but still predicates over an entity, and cases such as \emph{gemiddeld} `average', which could either predicate over an entity, as in \emph{gemiddelde student} `average student', or a predicate, as in \emph{zij eet gemiddeld 3 koekjes elke dag} `she eats in average 3 cookies per day'. Sometimes the incorrectly tagged cases (adverbs tagged as adjectives) were infrequent enough to be dismissed, but in some other cases they were so many we had to discard the lemma as candidate. While the only direct consequence is that a certain potentially interesting lemma couldn't be investigated, this also should be taken into account when relying on the part-of-speech tagger in other steps of the workflow.

The next subsection describes the selected nouns, adjectives and verbs, the QLVLNewsCorpus and the sampling method. For each selected item I will give an approximate equivalent in English and the definitions and examples as were provided to the annotators. A hypothesis of what the annotations for those items would look like will also be provided.

\hypertarget{nouns}{%
\subsection{The nouns}\label{nouns}}

The 8 nouns all exhibit homonymy and polysemy in at least one of the homonyms.

\begin{itemize}
\item
  Three nouns have one polysemous homonym and one non polysemous (\emph{hoop} `hope/bunch', \emph{spot} `ridicule/show or spotlight' and \emph{horde} `horde/hurdle');
\item
  four nouns have two polysemous homonyms (\emph{schaal} `scale/dish or shell', \emph{blik} `look/tin', \emph{stof} `substance or fabric or topic/dust', \emph{staal} `steal/sample');
\item
  one noun has three homonyms, of which two polysemous: \emph{spoor} `footprint or trace/train(line, rail, company)/spur'. This noun was later discarded because it proved too complicated, but the data is available for reanalysis.
\end{itemize}

For each group, the definitions, examples and expected relative frequency will be shown.

\hypertarget{one-polysemous-homonym}{%
\subsubsection{One polysemous homonym}\label{one-polysemous-homonym}}

The nouns in Table \ref{tab:nouns-1} have one frequent homonym and another one equally or less frequent and polysemic (so the sense distribution is probably very skewed). The polysemy phenomena represented by the polysemic homonym are different: metaphor for \emph{horde}, metaphor/generalization for \emph{hoop} and metonymy for \emph{spot}, where one of those senses (`spotlight') can be used literally or metaphorically (and annotators might either merge them or assign \emph{geen} to the figurative uses).

\providecommand{\docline}[3]{\noalign{\global\setlength{\arrayrulewidth}{#1}}\arrayrulecolor[HTML]{#2}\cline{#3}}

\setlength{\tabcolsep}{2pt}

\renewcommand*{\arraystretch}{1.5}

\begin{longtable}[c]{|p{0.75in}|p{0.75in}}

\caption{Some caption.}\label{tab:nouns-1}\\

\hhline{>{\arrayrulecolor[HTML]{666666}\global\arrayrulewidth=2pt}->{\arrayrulecolor[HTML]{666666}\global\arrayrulewidth=2pt}-}

\multicolumn{1}{!{\color[HTML]{000000}\vrule width 0pt}>{\raggedright}p{\dimexpr 0.75in+0\tabcolsep+0\arrayrulewidth}}{\fontsize{11}{11}\selectfont{\textcolor[HTML]{000000}{\global\setmainfont{Arial}lemma}}} & \multicolumn{1}{!{\color[HTML]{000000}\vrule width 0pt}>{\raggedright}p{\dimexpr 0.75in+0\tabcolsep+0\arrayrulewidth}!{\color[HTML]{000000}\vrule width 0pt}}{\fontsize{11}{11}\selectfont{\textcolor[HTML]{000000}{\global\setmainfont{Arial}text}}} \\

\noalign{\global\setlength{\arrayrulewidth}{2pt}}\arrayrulecolor[HTML]{666666}\cline{1-2}

\endfirsthead

\hhline{>{\arrayrulecolor[HTML]{666666}\global\arrayrulewidth=2pt}->{\arrayrulecolor[HTML]{666666}\global\arrayrulewidth=2pt}-}

\multicolumn{1}{!{\color[HTML]{000000}\vrule width 0pt}>{\raggedright}p{\dimexpr 0.75in+0\tabcolsep+0\arrayrulewidth}}{\fontsize{11}{11}\selectfont{\textcolor[HTML]{000000}{\global\setmainfont{Arial}lemma}}} & \multicolumn{1}{!{\color[HTML]{000000}\vrule width 0pt}>{\raggedright}p{\dimexpr 0.75in+0\tabcolsep+0\arrayrulewidth}!{\color[HTML]{000000}\vrule width 0pt}}{\fontsize{11}{11}\selectfont{\textcolor[HTML]{000000}{\global\setmainfont{Arial}text}}} \\

\noalign{\global\setlength{\arrayrulewidth}{2pt}}\arrayrulecolor[HTML]{666666}\cline{1-2}\endhead



\multicolumn{1}{!{\color[HTML]{000000}\vrule width 0pt}>{\raggedright}p{\dimexpr 0.75in+0\tabcolsep+0\arrayrulewidth}}{\fontsize{11}{11}\selectfont{\textcolor[HTML]{000000}{\global\setmainfont{Arial}hoop}}} & \multicolumn{1}{!{\color[HTML]{000000}\vrule width 0pt}>{\raggedright}p{\dimexpr 0.75in+0\tabcolsep+0\arrayrulewidth}!{\color[HTML]{000000}\vrule width 0pt}}{\fontsize{11}{11}\selectfont{\textcolor[HTML]{000000}{\global\setmainfont{Arial}Coming soon}}} \\





\multicolumn{1}{!{\color[HTML]{000000}\vrule width 0pt}>{\raggedright}p{\dimexpr 0.75in+0\tabcolsep+0\arrayrulewidth}}{\fontsize{11}{11}\selectfont{\textcolor[HTML]{000000}{\global\setmainfont{Arial}spot}}} & \multicolumn{1}{!{\color[HTML]{000000}\vrule width 0pt}>{\raggedright}p{\dimexpr 0.75in+0\tabcolsep+0\arrayrulewidth}!{\color[HTML]{000000}\vrule width 0pt}}{\fontsize{11}{11}\selectfont{\textcolor[HTML]{000000}{\global\setmainfont{Arial}Coming soon}}} \\





\multicolumn{1}{!{\color[HTML]{000000}\vrule width 0pt}>{\raggedright}p{\dimexpr 0.75in+0\tabcolsep+0\arrayrulewidth}}{\fontsize{11}{11}\selectfont{\textcolor[HTML]{000000}{\global\setmainfont{Arial}horde}}} & \multicolumn{1}{!{\color[HTML]{000000}\vrule width 0pt}>{\raggedright}p{\dimexpr 0.75in+0\tabcolsep+0\arrayrulewidth}!{\color[HTML]{000000}\vrule width 0pt}}{\fontsize{11}{11}\selectfont{\textcolor[HTML]{000000}{\global\setmainfont{Arial}Coming soon}}} \\

\noalign{\global\setlength{\arrayrulewidth}{2pt}}\arrayrulecolor[HTML]{666666}\cline{1-2}

\end{longtable}

\hypertarget{two-polysemous-homonyms}{%
\subsubsection{Two polysemous homonyms}\label{two-polysemous-homonyms}}

The homonyms in Table \ref{tab:nouns-2} also present a variety of polysemy phenomena. For \emph{blik}, the frequent homonym (`look') has a concrete sense with a metonymic and a metaphoric extension, while the infrequent one can refer to a material (`tin'), an object made of that material or its content: the distinction is quite clear but might depend on the specificity of the context and be very infrequent. Similarly, \emph{stof} presents one frequent homonym with two concrete, referentially distinct senses and an abstract one, and another with a subtle, context-specificity dependent difference. \emph{schaal} exhibits subtle perspective shifts in one homonym (`scale') and refers to different concrete objects with the second (`shell', `dish', `scale dish'). Finally, \emph{staal} `steal' could refer, like \emph{blik}, to either the material or an object made of it, while the `sample' homonym is sensitive to construal (focus on the use as evidence, when it's the case): it's likely to present high confusion and/or a very skewed distribution in both homonyms separately.

\providecommand{\docline}[3]{\noalign{\global\setlength{\arrayrulewidth}{#1}}\arrayrulecolor[HTML]{#2}\cline{#3}}

\setlength{\tabcolsep}{2pt}

\renewcommand*{\arraystretch}{1.5}

\begin{longtable}[c]{|p{0.75in}|p{0.75in}}

\caption{Some caption.}\label{tab:nouns-2}\\

\hhline{>{\arrayrulecolor[HTML]{666666}\global\arrayrulewidth=2pt}->{\arrayrulecolor[HTML]{666666}\global\arrayrulewidth=2pt}-}

\multicolumn{1}{!{\color[HTML]{000000}\vrule width 0pt}>{\raggedright}p{\dimexpr 0.75in+0\tabcolsep+0\arrayrulewidth}}{\fontsize{11}{11}\selectfont{\textcolor[HTML]{000000}{\global\setmainfont{Arial}lemma}}} & \multicolumn{1}{!{\color[HTML]{000000}\vrule width 0pt}>{\raggedright}p{\dimexpr 0.75in+0\tabcolsep+0\arrayrulewidth}!{\color[HTML]{000000}\vrule width 0pt}}{\fontsize{11}{11}\selectfont{\textcolor[HTML]{000000}{\global\setmainfont{Arial}text}}} \\

\noalign{\global\setlength{\arrayrulewidth}{2pt}}\arrayrulecolor[HTML]{666666}\cline{1-2}

\endfirsthead

\hhline{>{\arrayrulecolor[HTML]{666666}\global\arrayrulewidth=2pt}->{\arrayrulecolor[HTML]{666666}\global\arrayrulewidth=2pt}-}

\multicolumn{1}{!{\color[HTML]{000000}\vrule width 0pt}>{\raggedright}p{\dimexpr 0.75in+0\tabcolsep+0\arrayrulewidth}}{\fontsize{11}{11}\selectfont{\textcolor[HTML]{000000}{\global\setmainfont{Arial}lemma}}} & \multicolumn{1}{!{\color[HTML]{000000}\vrule width 0pt}>{\raggedright}p{\dimexpr 0.75in+0\tabcolsep+0\arrayrulewidth}!{\color[HTML]{000000}\vrule width 0pt}}{\fontsize{11}{11}\selectfont{\textcolor[HTML]{000000}{\global\setmainfont{Arial}text}}} \\

\noalign{\global\setlength{\arrayrulewidth}{2pt}}\arrayrulecolor[HTML]{666666}\cline{1-2}\endhead



\multicolumn{1}{!{\color[HTML]{000000}\vrule width 0pt}>{\raggedright}p{\dimexpr 0.75in+0\tabcolsep+0\arrayrulewidth}}{\fontsize{11}{11}\selectfont{\textcolor[HTML]{000000}{\global\setmainfont{Arial}schaal}}} & \multicolumn{1}{!{\color[HTML]{000000}\vrule width 0pt}>{\raggedright}p{\dimexpr 0.75in+0\tabcolsep+0\arrayrulewidth}!{\color[HTML]{000000}\vrule width 0pt}}{\fontsize{11}{11}\selectfont{\textcolor[HTML]{000000}{\global\setmainfont{Arial}Coming soon}}} \\





\multicolumn{1}{!{\color[HTML]{000000}\vrule width 0pt}>{\raggedright}p{\dimexpr 0.75in+0\tabcolsep+0\arrayrulewidth}}{\fontsize{11}{11}\selectfont{\textcolor[HTML]{000000}{\global\setmainfont{Arial}blik}}} & \multicolumn{1}{!{\color[HTML]{000000}\vrule width 0pt}>{\raggedright}p{\dimexpr 0.75in+0\tabcolsep+0\arrayrulewidth}!{\color[HTML]{000000}\vrule width 0pt}}{\fontsize{11}{11}\selectfont{\textcolor[HTML]{000000}{\global\setmainfont{Arial}Coming soon}}} \\





\multicolumn{1}{!{\color[HTML]{000000}\vrule width 0pt}>{\raggedright}p{\dimexpr 0.75in+0\tabcolsep+0\arrayrulewidth}}{\fontsize{11}{11}\selectfont{\textcolor[HTML]{000000}{\global\setmainfont{Arial}staal}}} & \multicolumn{1}{!{\color[HTML]{000000}\vrule width 0pt}>{\raggedright}p{\dimexpr 0.75in+0\tabcolsep+0\arrayrulewidth}!{\color[HTML]{000000}\vrule width 0pt}}{\fontsize{11}{11}\selectfont{\textcolor[HTML]{000000}{\global\setmainfont{Arial}Coming soon}}} \\





\multicolumn{1}{!{\color[HTML]{000000}\vrule width 0pt}>{\raggedright}p{\dimexpr 0.75in+0\tabcolsep+0\arrayrulewidth}}{\fontsize{11}{11}\selectfont{\textcolor[HTML]{000000}{\global\setmainfont{Arial}stof}}} & \multicolumn{1}{!{\color[HTML]{000000}\vrule width 0pt}>{\raggedright}p{\dimexpr 0.75in+0\tabcolsep+0\arrayrulewidth}!{\color[HTML]{000000}\vrule width 0pt}}{\fontsize{11}{11}\selectfont{\textcolor[HTML]{000000}{\global\setmainfont{Arial}Coming soon}}} \\

\noalign{\global\setlength{\arrayrulewidth}{2pt}}\arrayrulecolor[HTML]{666666}\cline{1-2}

\end{longtable}

\hypertarget{adjs}{%
\subsection{The adjectives}\label{adjs}}

The selection of adjectives includes 13 lemmas presenting different kinds of polysemy phenomena:

\begin{itemize}
\item
  three have a metonymic reading (\emph{hoopvol} `hopeful', \emph{geestig} `witty' and \emph{hachelijk} `dangerous/critical');
\item
  four have metaphoric readings (\emph{hoekig} `angulous/clumsy', \emph{dof} `dull', \emph{heilzaam} `healthy/beneficial' and \emph{gekleurd} `colorful, POC, tainted');
\item
  three present some other form of similarity between the readings (\emph{geldig} `valid', \emph{hemels} `heavenly' and \emph{gemeen} `shared/public/mean/serious');
\item
  three are more complex (\emph{heet} `hot' for different entities and metaphorically, \emph{grijs} `gray', with metaphorical and metonymical extensions and \emph{goedkoop} `cheap', with different entities and metaphorically).
\end{itemize}

For each group, the definitions and examples provided to the annotators will be shown with their estimated relative frequency and some considerations will be made regarding what we expect from the annotators.

\hypertarget{metonymic-cases}{%
\subsubsection{Metonymic cases}\label{metonymic-cases}}

As illustrated in Table \ref{tab:meton-adj}, for each of these three adjectives two senses were offered as options. For \emph{geestig} and \emph{hoopvol}, one of the senses is anthropocentric (it's mainly or exclusively applied to people), although such distinction is not made explicit in the definitions of \emph{geestig} (only suggested in the example). The expected frequency of the anthropomorphic sense is in both cases much higher than the other one. In \emph{hachelijk}'s case, the difference is a matter of temporal or telic perspective, so probably harder to distinguish, and it's probably more likely that annotators suggest the second sense as an alternative to the first one (assigning the `critical' interpretation to something potentially dangerous) than the other way around.

\providecommand{\docline}[3]{\noalign{\global\setlength{\arrayrulewidth}{#1}}\arrayrulecolor[HTML]{#2}\cline{#3}}

\setlength{\tabcolsep}{2pt}

\renewcommand*{\arraystretch}{1.5}

\begin{longtable}[c]{|p{0.75in}|p{0.75in}}

\caption{Some caption.}\label{tab:meton-adj}\\

\hhline{>{\arrayrulecolor[HTML]{666666}\global\arrayrulewidth=2pt}->{\arrayrulecolor[HTML]{666666}\global\arrayrulewidth=2pt}-}

\multicolumn{1}{!{\color[HTML]{000000}\vrule width 0pt}>{\raggedright}p{\dimexpr 0.75in+0\tabcolsep+0\arrayrulewidth}}{\fontsize{11}{11}\selectfont{\textcolor[HTML]{000000}{\global\setmainfont{Arial}lemma}}} & \multicolumn{1}{!{\color[HTML]{000000}\vrule width 0pt}>{\raggedright}p{\dimexpr 0.75in+0\tabcolsep+0\arrayrulewidth}!{\color[HTML]{000000}\vrule width 0pt}}{\fontsize{11}{11}\selectfont{\textcolor[HTML]{000000}{\global\setmainfont{Arial}text}}} \\

\noalign{\global\setlength{\arrayrulewidth}{2pt}}\arrayrulecolor[HTML]{666666}\cline{1-2}

\endfirsthead

\hhline{>{\arrayrulecolor[HTML]{666666}\global\arrayrulewidth=2pt}->{\arrayrulecolor[HTML]{666666}\global\arrayrulewidth=2pt}-}

\multicolumn{1}{!{\color[HTML]{000000}\vrule width 0pt}>{\raggedright}p{\dimexpr 0.75in+0\tabcolsep+0\arrayrulewidth}}{\fontsize{11}{11}\selectfont{\textcolor[HTML]{000000}{\global\setmainfont{Arial}lemma}}} & \multicolumn{1}{!{\color[HTML]{000000}\vrule width 0pt}>{\raggedright}p{\dimexpr 0.75in+0\tabcolsep+0\arrayrulewidth}!{\color[HTML]{000000}\vrule width 0pt}}{\fontsize{11}{11}\selectfont{\textcolor[HTML]{000000}{\global\setmainfont{Arial}text}}} \\

\noalign{\global\setlength{\arrayrulewidth}{2pt}}\arrayrulecolor[HTML]{666666}\cline{1-2}\endhead



\multicolumn{1}{!{\color[HTML]{000000}\vrule width 0pt}>{\raggedright}p{\dimexpr 0.75in+0\tabcolsep+0\arrayrulewidth}}{\fontsize{11}{11}\selectfont{\textcolor[HTML]{000000}{\global\setmainfont{Arial}hoopvol}}} & \multicolumn{1}{!{\color[HTML]{000000}\vrule width 0pt}>{\raggedright}p{\dimexpr 0.75in+0\tabcolsep+0\arrayrulewidth}!{\color[HTML]{000000}\vrule width 0pt}}{\fontsize{11}{11}\selectfont{\textcolor[HTML]{000000}{\global\setmainfont{Arial}Coming soon}}} \\





\multicolumn{1}{!{\color[HTML]{000000}\vrule width 0pt}>{\raggedright}p{\dimexpr 0.75in+0\tabcolsep+0\arrayrulewidth}}{\fontsize{11}{11}\selectfont{\textcolor[HTML]{000000}{\global\setmainfont{Arial}geestig}}} & \multicolumn{1}{!{\color[HTML]{000000}\vrule width 0pt}>{\raggedright}p{\dimexpr 0.75in+0\tabcolsep+0\arrayrulewidth}!{\color[HTML]{000000}\vrule width 0pt}}{\fontsize{11}{11}\selectfont{\textcolor[HTML]{000000}{\global\setmainfont{Arial}Coming soon}}} \\





\multicolumn{1}{!{\color[HTML]{000000}\vrule width 0pt}>{\raggedright}p{\dimexpr 0.75in+0\tabcolsep+0\arrayrulewidth}}{\fontsize{11}{11}\selectfont{\textcolor[HTML]{000000}{\global\setmainfont{Arial}hachelijk}}} & \multicolumn{1}{!{\color[HTML]{000000}\vrule width 0pt}>{\raggedright}p{\dimexpr 0.75in+0\tabcolsep+0\arrayrulewidth}!{\color[HTML]{000000}\vrule width 0pt}}{\fontsize{11}{11}\selectfont{\textcolor[HTML]{000000}{\global\setmainfont{Arial}Coming soon}}} \\

\noalign{\global\setlength{\arrayrulewidth}{2pt}}\arrayrulecolor[HTML]{666666}\cline{1-2}

\end{longtable}

\hypertarget{metaphoric-cases}{%
\subsubsection{Metaphoric cases}\label{metaphoric-cases}}

The adjectives with metaphoric extensions, presented in Table \ref{tab:metaf-adj}, have different numbers of senses. \emph{heilzaam} has two distinctions, between metaphoric and specialization: one refers to something specifically/literally healthy, and the other one is broader and less concrete. \emph{hoekig} and \emph{gekleurd} present three sense distinctions, one of which is particularly concrete and the most frequent and another one explicitly anthropocentric. The third sense distinction has a different quality: rather synesthetic for \emph{hoekig} and in an abstract, very much metaphoric for \emph{gekleurd}. Finally, \emph{dof} has all four kinds of senses: concrete, synesthetic, anthropocentric and abstract.

\providecommand{\docline}[3]{\noalign{\global\setlength{\arrayrulewidth}{#1}}\arrayrulecolor[HTML]{#2}\cline{#3}}

\setlength{\tabcolsep}{2pt}

\renewcommand*{\arraystretch}{1.5}

\begin{longtable}[c]{|p{0.75in}|p{0.75in}}

\caption{Some caption.}\label{tab:metaf-adj}\\

\hhline{>{\arrayrulecolor[HTML]{666666}\global\arrayrulewidth=2pt}->{\arrayrulecolor[HTML]{666666}\global\arrayrulewidth=2pt}-}

\multicolumn{1}{!{\color[HTML]{000000}\vrule width 0pt}>{\raggedright}p{\dimexpr 0.75in+0\tabcolsep+0\arrayrulewidth}}{\fontsize{11}{11}\selectfont{\textcolor[HTML]{000000}{\global\setmainfont{Arial}lemma}}} & \multicolumn{1}{!{\color[HTML]{000000}\vrule width 0pt}>{\raggedright}p{\dimexpr 0.75in+0\tabcolsep+0\arrayrulewidth}!{\color[HTML]{000000}\vrule width 0pt}}{\fontsize{11}{11}\selectfont{\textcolor[HTML]{000000}{\global\setmainfont{Arial}text}}} \\

\noalign{\global\setlength{\arrayrulewidth}{2pt}}\arrayrulecolor[HTML]{666666}\cline{1-2}

\endfirsthead

\hhline{>{\arrayrulecolor[HTML]{666666}\global\arrayrulewidth=2pt}->{\arrayrulecolor[HTML]{666666}\global\arrayrulewidth=2pt}-}

\multicolumn{1}{!{\color[HTML]{000000}\vrule width 0pt}>{\raggedright}p{\dimexpr 0.75in+0\tabcolsep+0\arrayrulewidth}}{\fontsize{11}{11}\selectfont{\textcolor[HTML]{000000}{\global\setmainfont{Arial}lemma}}} & \multicolumn{1}{!{\color[HTML]{000000}\vrule width 0pt}>{\raggedright}p{\dimexpr 0.75in+0\tabcolsep+0\arrayrulewidth}!{\color[HTML]{000000}\vrule width 0pt}}{\fontsize{11}{11}\selectfont{\textcolor[HTML]{000000}{\global\setmainfont{Arial}text}}} \\

\noalign{\global\setlength{\arrayrulewidth}{2pt}}\arrayrulecolor[HTML]{666666}\cline{1-2}\endhead



\multicolumn{1}{!{\color[HTML]{000000}\vrule width 0pt}>{\raggedright}p{\dimexpr 0.75in+0\tabcolsep+0\arrayrulewidth}}{\fontsize{11}{11}\selectfont{\textcolor[HTML]{000000}{\global\setmainfont{Arial}heilzaam}}} & \multicolumn{1}{!{\color[HTML]{000000}\vrule width 0pt}>{\raggedright}p{\dimexpr 0.75in+0\tabcolsep+0\arrayrulewidth}!{\color[HTML]{000000}\vrule width 0pt}}{\fontsize{11}{11}\selectfont{\textcolor[HTML]{000000}{\global\setmainfont{Arial}Coming soon}}} \\





\multicolumn{1}{!{\color[HTML]{000000}\vrule width 0pt}>{\raggedright}p{\dimexpr 0.75in+0\tabcolsep+0\arrayrulewidth}}{\fontsize{11}{11}\selectfont{\textcolor[HTML]{000000}{\global\setmainfont{Arial}gekleurd}}} & \multicolumn{1}{!{\color[HTML]{000000}\vrule width 0pt}>{\raggedright}p{\dimexpr 0.75in+0\tabcolsep+0\arrayrulewidth}!{\color[HTML]{000000}\vrule width 0pt}}{\fontsize{11}{11}\selectfont{\textcolor[HTML]{000000}{\global\setmainfont{Arial}Coming soon}}} \\





\multicolumn{1}{!{\color[HTML]{000000}\vrule width 0pt}>{\raggedright}p{\dimexpr 0.75in+0\tabcolsep+0\arrayrulewidth}}{\fontsize{11}{11}\selectfont{\textcolor[HTML]{000000}{\global\setmainfont{Arial}hoekig}}} & \multicolumn{1}{!{\color[HTML]{000000}\vrule width 0pt}>{\raggedright}p{\dimexpr 0.75in+0\tabcolsep+0\arrayrulewidth}!{\color[HTML]{000000}\vrule width 0pt}}{\fontsize{11}{11}\selectfont{\textcolor[HTML]{000000}{\global\setmainfont{Arial}Coming soon}}} \\





\multicolumn{1}{!{\color[HTML]{000000}\vrule width 0pt}>{\raggedright}p{\dimexpr 0.75in+0\tabcolsep+0\arrayrulewidth}}{\fontsize{11}{11}\selectfont{\textcolor[HTML]{000000}{\global\setmainfont{Arial}dof}}} & \multicolumn{1}{!{\color[HTML]{000000}\vrule width 0pt}>{\raggedright}p{\dimexpr 0.75in+0\tabcolsep+0\arrayrulewidth}!{\color[HTML]{000000}\vrule width 0pt}}{\fontsize{11}{11}\selectfont{\textcolor[HTML]{000000}{\global\setmainfont{Arial}Coming soon}}} \\

\noalign{\global\setlength{\arrayrulewidth}{2pt}}\arrayrulecolor[HTML]{666666}\cline{1-2}

\end{longtable}

\hypertarget{similarity}{%
\subsubsection{Similarity}\label{similarity}}

The group of adjectives in Table \ref{tab:simil-adj} present sense distinctions that could be roughly summarized under the title of `similarity', and are between generalization and shift of focus. \emph{geldig} and \emph{hemels} offer two options, one restricted to a specific context and one much broader. The relation between the relative frequencies of those senses are inverted (the specific sense of \emph{geldig} is less frequent than the general one, while in \emph{hemels} it's the other way around). We would expect that the specific sense would not be offered as alternative to the general sense as much as the other way around.

The case of \emph{gemeen} is quite complex, involving a number of rather subtle distinctions. The limits between the first and the second one and between the third and the fifth are hard to establish; the fourth sense seems more clear but if the context isn't specific enough it could be easily confused with the fifth. In addition, the senses are not always mutually exclusive, and a certain instance could very well conflate or be ambiguous between two senses.

\providecommand{\docline}[3]{\noalign{\global\setlength{\arrayrulewidth}{#1}}\arrayrulecolor[HTML]{#2}\cline{#3}}

\setlength{\tabcolsep}{2pt}

\renewcommand*{\arraystretch}{1.5}

\begin{longtable}[c]{|p{0.75in}|p{0.75in}}

\caption{Some caption.}\label{tab:simil-adj}\\

\hhline{>{\arrayrulecolor[HTML]{666666}\global\arrayrulewidth=2pt}->{\arrayrulecolor[HTML]{666666}\global\arrayrulewidth=2pt}-}

\multicolumn{1}{!{\color[HTML]{000000}\vrule width 0pt}>{\raggedright}p{\dimexpr 0.75in+0\tabcolsep+0\arrayrulewidth}}{\fontsize{11}{11}\selectfont{\textcolor[HTML]{000000}{\global\setmainfont{Arial}lemma}}} & \multicolumn{1}{!{\color[HTML]{000000}\vrule width 0pt}>{\raggedright}p{\dimexpr 0.75in+0\tabcolsep+0\arrayrulewidth}!{\color[HTML]{000000}\vrule width 0pt}}{\fontsize{11}{11}\selectfont{\textcolor[HTML]{000000}{\global\setmainfont{Arial}text}}} \\

\noalign{\global\setlength{\arrayrulewidth}{2pt}}\arrayrulecolor[HTML]{666666}\cline{1-2}

\endfirsthead

\hhline{>{\arrayrulecolor[HTML]{666666}\global\arrayrulewidth=2pt}->{\arrayrulecolor[HTML]{666666}\global\arrayrulewidth=2pt}-}

\multicolumn{1}{!{\color[HTML]{000000}\vrule width 0pt}>{\raggedright}p{\dimexpr 0.75in+0\tabcolsep+0\arrayrulewidth}}{\fontsize{11}{11}\selectfont{\textcolor[HTML]{000000}{\global\setmainfont{Arial}lemma}}} & \multicolumn{1}{!{\color[HTML]{000000}\vrule width 0pt}>{\raggedright}p{\dimexpr 0.75in+0\tabcolsep+0\arrayrulewidth}!{\color[HTML]{000000}\vrule width 0pt}}{\fontsize{11}{11}\selectfont{\textcolor[HTML]{000000}{\global\setmainfont{Arial}text}}} \\

\noalign{\global\setlength{\arrayrulewidth}{2pt}}\arrayrulecolor[HTML]{666666}\cline{1-2}\endhead



\multicolumn{1}{!{\color[HTML]{000000}\vrule width 0pt}>{\raggedright}p{\dimexpr 0.75in+0\tabcolsep+0\arrayrulewidth}}{\fontsize{11}{11}\selectfont{\textcolor[HTML]{000000}{\global\setmainfont{Arial}geldig}}} & \multicolumn{1}{!{\color[HTML]{000000}\vrule width 0pt}>{\raggedright}p{\dimexpr 0.75in+0\tabcolsep+0\arrayrulewidth}!{\color[HTML]{000000}\vrule width 0pt}}{\fontsize{11}{11}\selectfont{\textcolor[HTML]{000000}{\global\setmainfont{Arial}Coming soon}}} \\





\multicolumn{1}{!{\color[HTML]{000000}\vrule width 0pt}>{\raggedright}p{\dimexpr 0.75in+0\tabcolsep+0\arrayrulewidth}}{\fontsize{11}{11}\selectfont{\textcolor[HTML]{000000}{\global\setmainfont{Arial}hemels}}} & \multicolumn{1}{!{\color[HTML]{000000}\vrule width 0pt}>{\raggedright}p{\dimexpr 0.75in+0\tabcolsep+0\arrayrulewidth}!{\color[HTML]{000000}\vrule width 0pt}}{\fontsize{11}{11}\selectfont{\textcolor[HTML]{000000}{\global\setmainfont{Arial}Coming soon}}} \\





\multicolumn{1}{!{\color[HTML]{000000}\vrule width 0pt}>{\raggedright}p{\dimexpr 0.75in+0\tabcolsep+0\arrayrulewidth}}{\fontsize{11}{11}\selectfont{\textcolor[HTML]{000000}{\global\setmainfont{Arial}gemeen}}} & \multicolumn{1}{!{\color[HTML]{000000}\vrule width 0pt}>{\raggedright}p{\dimexpr 0.75in+0\tabcolsep+0\arrayrulewidth}!{\color[HTML]{000000}\vrule width 0pt}}{\fontsize{11}{11}\selectfont{\textcolor[HTML]{000000}{\global\setmainfont{Arial}Coming soon}}} \\

\noalign{\global\setlength{\arrayrulewidth}{2pt}}\arrayrulecolor[HTML]{666666}\cline{1-2}

\end{longtable}

\hypertarget{complex-cases}{%
\subsubsection{Complex cases}\label{complex-cases}}

The last group of adjectives, listed in Tables \ref{tab:heet}, \ref{tab:grijs} and \ref{tab:goedkoop}, have a large number of possible senses with more than one polysemy phenomenon, so they could be treated separately.

\emph{heet} presents three very concrete senses that differ in perspective (temperatures of different kinds of things). The second half is metaphorical, of which one synesthetic, one anthropocentric and very specific, and one more abstract and also quite specific. Furthermore, there is no exclusive sense tag for idiomatic expressions, which are quite frequent; they are expected to be tagged with the concrete senses (and maybe a comment on their figurative interpretation), but annotators might also use the \emph{geen} tag for those cases.

\providecommand{\docline}[3]{\noalign{\global\setlength{\arrayrulewidth}{#1}}\arrayrulecolor[HTML]{#2}\cline{#3}}

\setlength{\tabcolsep}{2pt}

\renewcommand*{\arraystretch}{1.5}

\begin{longtable}[c]{|p{0.75in}|p{0.75in}}

\caption{Some caption.}\label{tab:heet}\\

\hhline{>{\arrayrulecolor[HTML]{666666}\global\arrayrulewidth=2pt}->{\arrayrulecolor[HTML]{666666}\global\arrayrulewidth=2pt}-}

\multicolumn{1}{!{\color[HTML]{000000}\vrule width 0pt}>{\raggedright}p{\dimexpr 0.75in+0\tabcolsep+0\arrayrulewidth}}{\fontsize{11}{11}\selectfont{\textcolor[HTML]{000000}{\global\setmainfont{Arial}lemma}}} & \multicolumn{1}{!{\color[HTML]{000000}\vrule width 0pt}>{\raggedright}p{\dimexpr 0.75in+0\tabcolsep+0\arrayrulewidth}!{\color[HTML]{000000}\vrule width 0pt}}{\fontsize{11}{11}\selectfont{\textcolor[HTML]{000000}{\global\setmainfont{Arial}text}}} \\

\noalign{\global\setlength{\arrayrulewidth}{2pt}}\arrayrulecolor[HTML]{666666}\cline{1-2}

\endfirsthead

\hhline{>{\arrayrulecolor[HTML]{666666}\global\arrayrulewidth=2pt}->{\arrayrulecolor[HTML]{666666}\global\arrayrulewidth=2pt}-}

\multicolumn{1}{!{\color[HTML]{000000}\vrule width 0pt}>{\raggedright}p{\dimexpr 0.75in+0\tabcolsep+0\arrayrulewidth}}{\fontsize{11}{11}\selectfont{\textcolor[HTML]{000000}{\global\setmainfont{Arial}lemma}}} & \multicolumn{1}{!{\color[HTML]{000000}\vrule width 0pt}>{\raggedright}p{\dimexpr 0.75in+0\tabcolsep+0\arrayrulewidth}!{\color[HTML]{000000}\vrule width 0pt}}{\fontsize{11}{11}\selectfont{\textcolor[HTML]{000000}{\global\setmainfont{Arial}text}}} \\

\noalign{\global\setlength{\arrayrulewidth}{2pt}}\arrayrulecolor[HTML]{666666}\cline{1-2}\endhead



\multicolumn{1}{!{\color[HTML]{000000}\vrule width 0pt}>{\raggedright}p{\dimexpr 0.75in+0\tabcolsep+0\arrayrulewidth}}{\fontsize{11}{11}\selectfont{\textcolor[HTML]{000000}{\global\setmainfont{Arial}heet}}} & \multicolumn{1}{!{\color[HTML]{000000}\vrule width 0pt}>{\raggedright}p{\dimexpr 0.75in+0\tabcolsep+0\arrayrulewidth}!{\color[HTML]{000000}\vrule width 0pt}}{\fontsize{11}{11}\selectfont{\textcolor[HTML]{000000}{\global\setmainfont{Arial}Coming soon}}} \\

\noalign{\global\setlength{\arrayrulewidth}{2pt}}\arrayrulecolor[HTML]{666666}\cline{1-2}

\end{longtable}

\emph{grijs} presents a very frequent, concrete sense, two specific metonymic extensions, one anthropocentric sense, one rather abstract and another very specific metaphor.

\providecommand{\docline}[3]{\noalign{\global\setlength{\arrayrulewidth}{#1}}\arrayrulecolor[HTML]{#2}\cline{#3}}

\setlength{\tabcolsep}{2pt}

\renewcommand*{\arraystretch}{1.5}

\begin{longtable}[c]{|p{0.75in}|p{0.75in}}

\caption{Some caption.}\label{tab:grijs}\\

\hhline{>{\arrayrulecolor[HTML]{666666}\global\arrayrulewidth=2pt}->{\arrayrulecolor[HTML]{666666}\global\arrayrulewidth=2pt}-}

\multicolumn{1}{!{\color[HTML]{000000}\vrule width 0pt}>{\raggedright}p{\dimexpr 0.75in+0\tabcolsep+0\arrayrulewidth}}{\fontsize{11}{11}\selectfont{\textcolor[HTML]{000000}{\global\setmainfont{Arial}lemma}}} & \multicolumn{1}{!{\color[HTML]{000000}\vrule width 0pt}>{\raggedright}p{\dimexpr 0.75in+0\tabcolsep+0\arrayrulewidth}!{\color[HTML]{000000}\vrule width 0pt}}{\fontsize{11}{11}\selectfont{\textcolor[HTML]{000000}{\global\setmainfont{Arial}text}}} \\

\noalign{\global\setlength{\arrayrulewidth}{2pt}}\arrayrulecolor[HTML]{666666}\cline{1-2}

\endfirsthead

\hhline{>{\arrayrulecolor[HTML]{666666}\global\arrayrulewidth=2pt}->{\arrayrulecolor[HTML]{666666}\global\arrayrulewidth=2pt}-}

\multicolumn{1}{!{\color[HTML]{000000}\vrule width 0pt}>{\raggedright}p{\dimexpr 0.75in+0\tabcolsep+0\arrayrulewidth}}{\fontsize{11}{11}\selectfont{\textcolor[HTML]{000000}{\global\setmainfont{Arial}lemma}}} & \multicolumn{1}{!{\color[HTML]{000000}\vrule width 0pt}>{\raggedright}p{\dimexpr 0.75in+0\tabcolsep+0\arrayrulewidth}!{\color[HTML]{000000}\vrule width 0pt}}{\fontsize{11}{11}\selectfont{\textcolor[HTML]{000000}{\global\setmainfont{Arial}text}}} \\

\noalign{\global\setlength{\arrayrulewidth}{2pt}}\arrayrulecolor[HTML]{666666}\cline{1-2}\endhead



\multicolumn{1}{!{\color[HTML]{000000}\vrule width 0pt}>{\raggedright}p{\dimexpr 0.75in+0\tabcolsep+0\arrayrulewidth}}{\fontsize{11}{11}\selectfont{\textcolor[HTML]{000000}{\global\setmainfont{Arial}grijs}}} & \multicolumn{1}{!{\color[HTML]{000000}\vrule width 0pt}>{\raggedright}p{\dimexpr 0.75in+0\tabcolsep+0\arrayrulewidth}!{\color[HTML]{000000}\vrule width 0pt}}{\fontsize{11}{11}\selectfont{\textcolor[HTML]{000000}{\global\setmainfont{Arial}Coming soon}}} \\

\noalign{\global\setlength{\arrayrulewidth}{2pt}}\arrayrulecolor[HTML]{666666}\cline{1-2}

\end{longtable}

\emph{goedkoop}, on the other hand, presents ``only'' 4 sense distinctions: a concrete, prototypical and frequent sense, two perspectival shifts and a clear metaphor.

\providecommand{\docline}[3]{\noalign{\global\setlength{\arrayrulewidth}{#1}}\arrayrulecolor[HTML]{#2}\cline{#3}}

\setlength{\tabcolsep}{2pt}

\renewcommand*{\arraystretch}{1.5}

\begin{longtable}[c]{|p{0.75in}|p{0.75in}}

\caption{Some caption.}\label{tab:goedkoop}\\

\hhline{>{\arrayrulecolor[HTML]{666666}\global\arrayrulewidth=2pt}->{\arrayrulecolor[HTML]{666666}\global\arrayrulewidth=2pt}-}

\multicolumn{1}{!{\color[HTML]{000000}\vrule width 0pt}>{\raggedright}p{\dimexpr 0.75in+0\tabcolsep+0\arrayrulewidth}}{\fontsize{11}{11}\selectfont{\textcolor[HTML]{000000}{\global\setmainfont{Arial}lemma}}} & \multicolumn{1}{!{\color[HTML]{000000}\vrule width 0pt}>{\raggedright}p{\dimexpr 0.75in+0\tabcolsep+0\arrayrulewidth}!{\color[HTML]{000000}\vrule width 0pt}}{\fontsize{11}{11}\selectfont{\textcolor[HTML]{000000}{\global\setmainfont{Arial}text}}} \\

\noalign{\global\setlength{\arrayrulewidth}{2pt}}\arrayrulecolor[HTML]{666666}\cline{1-2}

\endfirsthead

\hhline{>{\arrayrulecolor[HTML]{666666}\global\arrayrulewidth=2pt}->{\arrayrulecolor[HTML]{666666}\global\arrayrulewidth=2pt}-}

\multicolumn{1}{!{\color[HTML]{000000}\vrule width 0pt}>{\raggedright}p{\dimexpr 0.75in+0\tabcolsep+0\arrayrulewidth}}{\fontsize{11}{11}\selectfont{\textcolor[HTML]{000000}{\global\setmainfont{Arial}lemma}}} & \multicolumn{1}{!{\color[HTML]{000000}\vrule width 0pt}>{\raggedright}p{\dimexpr 0.75in+0\tabcolsep+0\arrayrulewidth}!{\color[HTML]{000000}\vrule width 0pt}}{\fontsize{11}{11}\selectfont{\textcolor[HTML]{000000}{\global\setmainfont{Arial}text}}} \\

\noalign{\global\setlength{\arrayrulewidth}{2pt}}\arrayrulecolor[HTML]{666666}\cline{1-2}\endhead



\multicolumn{1}{!{\color[HTML]{000000}\vrule width 0pt}>{\raggedright}p{\dimexpr 0.75in+0\tabcolsep+0\arrayrulewidth}}{\fontsize{11}{11}\selectfont{\textcolor[HTML]{000000}{\global\setmainfont{Arial}goedkoop}}} & \multicolumn{1}{!{\color[HTML]{000000}\vrule width 0pt}>{\raggedright}p{\dimexpr 0.75in+0\tabcolsep+0\arrayrulewidth}!{\color[HTML]{000000}\vrule width 0pt}}{\fontsize{11}{11}\selectfont{\textcolor[HTML]{000000}{\global\setmainfont{Arial}Coming soon}}} \\

\noalign{\global\setlength{\arrayrulewidth}{2pt}}\arrayrulecolor[HTML]{666666}\cline{1-2}

\end{longtable}

\hypertarget{verbs}{%
\subsection{The verbs}\label{verbs}}

For the verbs, we selected a range of combinations of syntactic and semantic variation:

\begin{itemize}
\item
  Transitive verbs where the sense distinction is related to the objects it can take (\emph{haten} `hate', \emph{huldigen} `honor/hold (attitudes, opinions, stances)', \emph{heffen} `raise', \emph{herroepen} `annul (a law)/retract (statement)');
\item
  Verbs that could be transitive, with a distinction based on the object, or intranstive (\emph{helpen} `help', \emph{herstructureren} `restructure');
\item
  Verbs that could be transitive, with a distinction based on the object, or reflexive (\emph{diskwalificeren} `disqualify', \emph{herhalen} `repeat', \emph{herinneren} `remember/remind', \emph{herkennen} `recognize');
\item
  Verbs that could be transitive, intransitive or reflexive, with semantic distinctions within the transtive structure (\emph{harden} `make/become hard, tolerate', \emph{herstellen} `heal/repair');
\item
  A verb with semantic distinctions within both the transitive and the intransitive structures (but also other problems): \emph{haken} `hook' (literally or metaphorically), `crochet', `make someone trip' (when the object is a person or \emph{pootje} `leg'), `get stuck' in the intransitive form as \emph{blijven haken} (literally or metaphorically).
\end{itemize}

For each of the groups, the definitions, examples and relative frequencies of the senses will be summarized in tables.

\hypertarget{only-transitive}{%
\subsubsection{Only transitive}\label{only-transitive}}

The verbs in the first group (Table \ref{tab:tr-verbs}) are always transitive and their different senses correlate with the possible direct objects they could take. \emph{haten} and \emph{heffen} are probably more easy to distinguish, the former having an anthropocentric distinction (basically, hating people against disliking things, but with possible gray areas in between, depending on how that object is construed) and the latter presenting a rather clear and common metaphor, between physical objects and abstract entities such as taxes being raised. \emph{huldigen} and \emph{herroepen} instead have slightly more subtle differences, but the former (between honoring someone/something and holding and opinion) is probably stronger and easier to distinguish than the latter, between retracting a statement or annuling a decree (which again could be interpreted differently depending on how the entity is construed, how prototypical it is).

\providecommand{\docline}[3]{\noalign{\global\setlength{\arrayrulewidth}{#1}}\arrayrulecolor[HTML]{#2}\cline{#3}}

\setlength{\tabcolsep}{2pt}

\renewcommand*{\arraystretch}{1.5}

\begin{longtable}[c]{|p{0.75in}|p{0.75in}}

\caption{Some caption.}\label{tab:tr-verbs}\\

\hhline{>{\arrayrulecolor[HTML]{666666}\global\arrayrulewidth=2pt}->{\arrayrulecolor[HTML]{666666}\global\arrayrulewidth=2pt}-}

\multicolumn{1}{!{\color[HTML]{000000}\vrule width 0pt}>{\raggedright}p{\dimexpr 0.75in+0\tabcolsep+0\arrayrulewidth}}{\fontsize{11}{11}\selectfont{\textcolor[HTML]{000000}{\global\setmainfont{Arial}lemma}}} & \multicolumn{1}{!{\color[HTML]{000000}\vrule width 0pt}>{\raggedright}p{\dimexpr 0.75in+0\tabcolsep+0\arrayrulewidth}!{\color[HTML]{000000}\vrule width 0pt}}{\fontsize{11}{11}\selectfont{\textcolor[HTML]{000000}{\global\setmainfont{Arial}text}}} \\

\noalign{\global\setlength{\arrayrulewidth}{2pt}}\arrayrulecolor[HTML]{666666}\cline{1-2}

\endfirsthead

\hhline{>{\arrayrulecolor[HTML]{666666}\global\arrayrulewidth=2pt}->{\arrayrulecolor[HTML]{666666}\global\arrayrulewidth=2pt}-}

\multicolumn{1}{!{\color[HTML]{000000}\vrule width 0pt}>{\raggedright}p{\dimexpr 0.75in+0\tabcolsep+0\arrayrulewidth}}{\fontsize{11}{11}\selectfont{\textcolor[HTML]{000000}{\global\setmainfont{Arial}lemma}}} & \multicolumn{1}{!{\color[HTML]{000000}\vrule width 0pt}>{\raggedright}p{\dimexpr 0.75in+0\tabcolsep+0\arrayrulewidth}!{\color[HTML]{000000}\vrule width 0pt}}{\fontsize{11}{11}\selectfont{\textcolor[HTML]{000000}{\global\setmainfont{Arial}text}}} \\

\noalign{\global\setlength{\arrayrulewidth}{2pt}}\arrayrulecolor[HTML]{666666}\cline{1-2}\endhead



\multicolumn{1}{!{\color[HTML]{000000}\vrule width 0pt}>{\raggedright}p{\dimexpr 0.75in+0\tabcolsep+0\arrayrulewidth}}{\fontsize{11}{11}\selectfont{\textcolor[HTML]{000000}{\global\setmainfont{Arial}haten}}} & \multicolumn{1}{!{\color[HTML]{000000}\vrule width 0pt}>{\raggedright}p{\dimexpr 0.75in+0\tabcolsep+0\arrayrulewidth}!{\color[HTML]{000000}\vrule width 0pt}}{\fontsize{11}{11}\selectfont{\textcolor[HTML]{000000}{\global\setmainfont{Arial}Coming soon}}} \\





\multicolumn{1}{!{\color[HTML]{000000}\vrule width 0pt}>{\raggedright}p{\dimexpr 0.75in+0\tabcolsep+0\arrayrulewidth}}{\fontsize{11}{11}\selectfont{\textcolor[HTML]{000000}{\global\setmainfont{Arial}huldigen}}} & \multicolumn{1}{!{\color[HTML]{000000}\vrule width 0pt}>{\raggedright}p{\dimexpr 0.75in+0\tabcolsep+0\arrayrulewidth}!{\color[HTML]{000000}\vrule width 0pt}}{\fontsize{11}{11}\selectfont{\textcolor[HTML]{000000}{\global\setmainfont{Arial}Coming soon}}} \\





\multicolumn{1}{!{\color[HTML]{000000}\vrule width 0pt}>{\raggedright}p{\dimexpr 0.75in+0\tabcolsep+0\arrayrulewidth}}{\fontsize{11}{11}\selectfont{\textcolor[HTML]{000000}{\global\setmainfont{Arial}heffen}}} & \multicolumn{1}{!{\color[HTML]{000000}\vrule width 0pt}>{\raggedright}p{\dimexpr 0.75in+0\tabcolsep+0\arrayrulewidth}!{\color[HTML]{000000}\vrule width 0pt}}{\fontsize{11}{11}\selectfont{\textcolor[HTML]{000000}{\global\setmainfont{Arial}Coming soon}}} \\





\multicolumn{1}{!{\color[HTML]{000000}\vrule width 0pt}>{\raggedright}p{\dimexpr 0.75in+0\tabcolsep+0\arrayrulewidth}}{\fontsize{11}{11}\selectfont{\textcolor[HTML]{000000}{\global\setmainfont{Arial}herroepen}}} & \multicolumn{1}{!{\color[HTML]{000000}\vrule width 0pt}>{\raggedright}p{\dimexpr 0.75in+0\tabcolsep+0\arrayrulewidth}!{\color[HTML]{000000}\vrule width 0pt}}{\fontsize{11}{11}\selectfont{\textcolor[HTML]{000000}{\global\setmainfont{Arial}Coming soon}}} \\

\noalign{\global\setlength{\arrayrulewidth}{2pt}}\arrayrulecolor[HTML]{666666}\cline{1-2}

\end{longtable}

\hypertarget{transitive-and-intransitive}{%
\subsubsection{Transitive and intransitive}\label{transitive-and-intransitive}}

The verbs that alternate between transitive and intransitive constructions (Table \ref{tab:tr2a-verbs}) are quite subtle and might present a lot of confusion, particularly because the intransitive cases are very similar to one of the transitive cases, and it might seem that the object is just ellided. For \emph{herstructureren}, one transitive sense and the intransitive one (exemplified with a reflexive\ldots) are more specific, regarding companies and with the connotation that the personnel is being reduced, while the other transitive sense is broader and might be selected in contexts with less specificity. It might also depend on world knowledge (whether the annotators know or can guess that a certain object -or the subject in the intransitive construction- is a company) and how prominent the implication of personnel reduction is. For \emph{helpen}, the distinction between the transitive uses is rather subtle (the ``collaboration'' sense is exclusive of animate subjects, but that's not explicit in the definitions), so there might be some disagreement in their annotation, but if the intransitive sense is confused with the transitive ones, it should only be with the first one.

\providecommand{\docline}[3]{\noalign{\global\setlength{\arrayrulewidth}{#1}}\arrayrulecolor[HTML]{#2}\cline{#3}}

\setlength{\tabcolsep}{2pt}

\renewcommand*{\arraystretch}{1.5}

\begin{longtable}[c]{|p{0.75in}|p{0.75in}}

\caption{Some caption.}\label{tab:tr2a-verbs}\\

\hhline{>{\arrayrulecolor[HTML]{666666}\global\arrayrulewidth=2pt}->{\arrayrulecolor[HTML]{666666}\global\arrayrulewidth=2pt}-}

\multicolumn{1}{!{\color[HTML]{000000}\vrule width 0pt}>{\raggedright}p{\dimexpr 0.75in+0\tabcolsep+0\arrayrulewidth}}{\fontsize{11}{11}\selectfont{\textcolor[HTML]{000000}{\global\setmainfont{Arial}lemma}}} & \multicolumn{1}{!{\color[HTML]{000000}\vrule width 0pt}>{\raggedright}p{\dimexpr 0.75in+0\tabcolsep+0\arrayrulewidth}!{\color[HTML]{000000}\vrule width 0pt}}{\fontsize{11}{11}\selectfont{\textcolor[HTML]{000000}{\global\setmainfont{Arial}text}}} \\

\noalign{\global\setlength{\arrayrulewidth}{2pt}}\arrayrulecolor[HTML]{666666}\cline{1-2}

\endfirsthead

\hhline{>{\arrayrulecolor[HTML]{666666}\global\arrayrulewidth=2pt}->{\arrayrulecolor[HTML]{666666}\global\arrayrulewidth=2pt}-}

\multicolumn{1}{!{\color[HTML]{000000}\vrule width 0pt}>{\raggedright}p{\dimexpr 0.75in+0\tabcolsep+0\arrayrulewidth}}{\fontsize{11}{11}\selectfont{\textcolor[HTML]{000000}{\global\setmainfont{Arial}lemma}}} & \multicolumn{1}{!{\color[HTML]{000000}\vrule width 0pt}>{\raggedright}p{\dimexpr 0.75in+0\tabcolsep+0\arrayrulewidth}!{\color[HTML]{000000}\vrule width 0pt}}{\fontsize{11}{11}\selectfont{\textcolor[HTML]{000000}{\global\setmainfont{Arial}text}}} \\

\noalign{\global\setlength{\arrayrulewidth}{2pt}}\arrayrulecolor[HTML]{666666}\cline{1-2}\endhead



\multicolumn{1}{!{\color[HTML]{000000}\vrule width 0pt}>{\raggedright}p{\dimexpr 0.75in+0\tabcolsep+0\arrayrulewidth}}{\fontsize{11}{11}\selectfont{\textcolor[HTML]{000000}{\global\setmainfont{Arial}helpen}}} & \multicolumn{1}{!{\color[HTML]{000000}\vrule width 0pt}>{\raggedright}p{\dimexpr 0.75in+0\tabcolsep+0\arrayrulewidth}!{\color[HTML]{000000}\vrule width 0pt}}{\fontsize{11}{11}\selectfont{\textcolor[HTML]{000000}{\global\setmainfont{Arial}Coming soon}}} \\





\multicolumn{1}{!{\color[HTML]{000000}\vrule width 0pt}>{\raggedright}p{\dimexpr 0.75in+0\tabcolsep+0\arrayrulewidth}}{\fontsize{11}{11}\selectfont{\textcolor[HTML]{000000}{\global\setmainfont{Arial}herstructureren}}} & \multicolumn{1}{!{\color[HTML]{000000}\vrule width 0pt}>{\raggedright}p{\dimexpr 0.75in+0\tabcolsep+0\arrayrulewidth}!{\color[HTML]{000000}\vrule width 0pt}}{\fontsize{11}{11}\selectfont{\textcolor[HTML]{000000}{\global\setmainfont{Arial}Coming soon}}} \\

\noalign{\global\setlength{\arrayrulewidth}{2pt}}\arrayrulecolor[HTML]{666666}\cline{1-2}

\end{longtable}

\hypertarget{transitive-and-reflexive}{%
\subsubsection{Transitive and reflexive}\label{transitive-and-reflexive}}

It could be argued that the distinction between the argument structures in the previous group (transitive versus intransitive) is more clear than that in this group (between transitive and reflexive), since the reflexive pronoun \emph{could} in some contexts be interpreted as a direct object. However, the intransitive senses in the previous group could also be understood as transitive cases with the object outside of context (true?), while the reflexive argument structure here (Table \ref{tab:tr2b-verbs}) is more strongly distinguishable from the transitive one. It might be tricky with \emph{diskwalificeren}, where the reflexive argument structure pretty much replicates the transitive senses, in a particular case where someone disqualifies themselves. The possibility to distinguish between the transitive cases, which differ in specificity, relies instead on the clarity of the context. For \emph{herkennen}, the sense distinction between the three transitive uses is quite subtle, and much sharper between transitive and reflexive; for \emph{herhalen}, what could be an object in the transitive senses (but probably wouldn't) is the subject in the reflexive, so the distinction should be very clear, while the transitive uses differ in the kind of objects that they take, with certain prototypical nouns (and the possibility of clauses for the second sense) and maybe some borderline cases. Finally, \emph{herinneren} shows both a clear distinction between reflexive and transitive uses and a further argument structure distinction between transitive uses, either with or without an \emph{aan} complement (which might be absent in the restricted context).

\providecommand{\docline}[3]{\noalign{\global\setlength{\arrayrulewidth}{#1}}\arrayrulecolor[HTML]{#2}\cline{#3}}

\setlength{\tabcolsep}{2pt}

\renewcommand*{\arraystretch}{1.5}

\begin{longtable}[c]{|p{0.75in}|p{0.75in}}

\caption{Some caption.}\label{tab:tr2b-verbs}\\

\hhline{>{\arrayrulecolor[HTML]{666666}\global\arrayrulewidth=2pt}->{\arrayrulecolor[HTML]{666666}\global\arrayrulewidth=2pt}-}

\multicolumn{1}{!{\color[HTML]{000000}\vrule width 0pt}>{\raggedright}p{\dimexpr 0.75in+0\tabcolsep+0\arrayrulewidth}}{\fontsize{11}{11}\selectfont{\textcolor[HTML]{000000}{\global\setmainfont{Arial}lemma}}} & \multicolumn{1}{!{\color[HTML]{000000}\vrule width 0pt}>{\raggedright}p{\dimexpr 0.75in+0\tabcolsep+0\arrayrulewidth}!{\color[HTML]{000000}\vrule width 0pt}}{\fontsize{11}{11}\selectfont{\textcolor[HTML]{000000}{\global\setmainfont{Arial}text}}} \\

\noalign{\global\setlength{\arrayrulewidth}{2pt}}\arrayrulecolor[HTML]{666666}\cline{1-2}

\endfirsthead

\hhline{>{\arrayrulecolor[HTML]{666666}\global\arrayrulewidth=2pt}->{\arrayrulecolor[HTML]{666666}\global\arrayrulewidth=2pt}-}

\multicolumn{1}{!{\color[HTML]{000000}\vrule width 0pt}>{\raggedright}p{\dimexpr 0.75in+0\tabcolsep+0\arrayrulewidth}}{\fontsize{11}{11}\selectfont{\textcolor[HTML]{000000}{\global\setmainfont{Arial}lemma}}} & \multicolumn{1}{!{\color[HTML]{000000}\vrule width 0pt}>{\raggedright}p{\dimexpr 0.75in+0\tabcolsep+0\arrayrulewidth}!{\color[HTML]{000000}\vrule width 0pt}}{\fontsize{11}{11}\selectfont{\textcolor[HTML]{000000}{\global\setmainfont{Arial}text}}} \\

\noalign{\global\setlength{\arrayrulewidth}{2pt}}\arrayrulecolor[HTML]{666666}\cline{1-2}\endhead



\multicolumn{1}{!{\color[HTML]{000000}\vrule width 0pt}>{\raggedright}p{\dimexpr 0.75in+0\tabcolsep+0\arrayrulewidth}}{\fontsize{11}{11}\selectfont{\textcolor[HTML]{000000}{\global\setmainfont{Arial}diskwalificeren}}} & \multicolumn{1}{!{\color[HTML]{000000}\vrule width 0pt}>{\raggedright}p{\dimexpr 0.75in+0\tabcolsep+0\arrayrulewidth}!{\color[HTML]{000000}\vrule width 0pt}}{\fontsize{11}{11}\selectfont{\textcolor[HTML]{000000}{\global\setmainfont{Arial}Coming soon}}} \\





\multicolumn{1}{!{\color[HTML]{000000}\vrule width 0pt}>{\raggedright}p{\dimexpr 0.75in+0\tabcolsep+0\arrayrulewidth}}{\fontsize{11}{11}\selectfont{\textcolor[HTML]{000000}{\global\setmainfont{Arial}herhalen}}} & \multicolumn{1}{!{\color[HTML]{000000}\vrule width 0pt}>{\raggedright}p{\dimexpr 0.75in+0\tabcolsep+0\arrayrulewidth}!{\color[HTML]{000000}\vrule width 0pt}}{\fontsize{11}{11}\selectfont{\textcolor[HTML]{000000}{\global\setmainfont{Arial}Coming soon}}} \\





\multicolumn{1}{!{\color[HTML]{000000}\vrule width 0pt}>{\raggedright}p{\dimexpr 0.75in+0\tabcolsep+0\arrayrulewidth}}{\fontsize{11}{11}\selectfont{\textcolor[HTML]{000000}{\global\setmainfont{Arial}herinneren}}} & \multicolumn{1}{!{\color[HTML]{000000}\vrule width 0pt}>{\raggedright}p{\dimexpr 0.75in+0\tabcolsep+0\arrayrulewidth}!{\color[HTML]{000000}\vrule width 0pt}}{\fontsize{11}{11}\selectfont{\textcolor[HTML]{000000}{\global\setmainfont{Arial}Coming soon}}} \\





\multicolumn{1}{!{\color[HTML]{000000}\vrule width 0pt}>{\raggedright}p{\dimexpr 0.75in+0\tabcolsep+0\arrayrulewidth}}{\fontsize{11}{11}\selectfont{\textcolor[HTML]{000000}{\global\setmainfont{Arial}herkennen}}} & \multicolumn{1}{!{\color[HTML]{000000}\vrule width 0pt}>{\raggedright}p{\dimexpr 0.75in+0\tabcolsep+0\arrayrulewidth}!{\color[HTML]{000000}\vrule width 0pt}}{\fontsize{11}{11}\selectfont{\textcolor[HTML]{000000}{\global\setmainfont{Arial}Coming soon}}} \\

\noalign{\global\setlength{\arrayrulewidth}{2pt}}\arrayrulecolor[HTML]{666666}\cline{1-2}

\end{longtable}

\hypertarget{transitive-intransitive-and-reflexive}{%
\subsubsection{Transitive, intransitive and reflexive}\label{transitive-intransitive-and-reflexive}}

The verbs in Table \ref{tab:tr3-verbs} can occur in the three argument structures: transitive (with three different senses), intransitive and reflexive. In the case of \emph{harden}, the transitive can be concrete, figurative, or concrete with a different sense and in a specific construction, namely \emph{(niet) te harden}; the intransitive structure is similar to the concrete transitive, but taking its object as subject, and the reflexive is similar to the second transitive. If senses of different argument structures were confused, the intransitive would be with the first and the reflexive with the second. The \emph{(niet) te harden} uses should be easy to isolate, with strong agreement between annotators and high confidence. For \emph{herstellen}, the transitive structure presents three possible senses: one concrete, one figurative but not presented as such, and one abstract that is very subtly different from the second one. The reflexive is very close to the figurative sense and the intransitive is more specific to concrete healing (rather than repairing) and should not be confused with the others.

\providecommand{\docline}[3]{\noalign{\global\setlength{\arrayrulewidth}{#1}}\arrayrulecolor[HTML]{#2}\cline{#3}}

\setlength{\tabcolsep}{2pt}

\renewcommand*{\arraystretch}{1.5}

\begin{longtable}[c]{|p{0.75in}|p{0.75in}}

\caption{Some caption.}\label{tab:tr3-verbs}\\

\hhline{>{\arrayrulecolor[HTML]{666666}\global\arrayrulewidth=2pt}->{\arrayrulecolor[HTML]{666666}\global\arrayrulewidth=2pt}-}

\multicolumn{1}{!{\color[HTML]{000000}\vrule width 0pt}>{\raggedright}p{\dimexpr 0.75in+0\tabcolsep+0\arrayrulewidth}}{\fontsize{11}{11}\selectfont{\textcolor[HTML]{000000}{\global\setmainfont{Arial}lemma}}} & \multicolumn{1}{!{\color[HTML]{000000}\vrule width 0pt}>{\raggedright}p{\dimexpr 0.75in+0\tabcolsep+0\arrayrulewidth}!{\color[HTML]{000000}\vrule width 0pt}}{\fontsize{11}{11}\selectfont{\textcolor[HTML]{000000}{\global\setmainfont{Arial}text}}} \\

\noalign{\global\setlength{\arrayrulewidth}{2pt}}\arrayrulecolor[HTML]{666666}\cline{1-2}

\endfirsthead

\hhline{>{\arrayrulecolor[HTML]{666666}\global\arrayrulewidth=2pt}->{\arrayrulecolor[HTML]{666666}\global\arrayrulewidth=2pt}-}

\multicolumn{1}{!{\color[HTML]{000000}\vrule width 0pt}>{\raggedright}p{\dimexpr 0.75in+0\tabcolsep+0\arrayrulewidth}}{\fontsize{11}{11}\selectfont{\textcolor[HTML]{000000}{\global\setmainfont{Arial}lemma}}} & \multicolumn{1}{!{\color[HTML]{000000}\vrule width 0pt}>{\raggedright}p{\dimexpr 0.75in+0\tabcolsep+0\arrayrulewidth}!{\color[HTML]{000000}\vrule width 0pt}}{\fontsize{11}{11}\selectfont{\textcolor[HTML]{000000}{\global\setmainfont{Arial}text}}} \\

\noalign{\global\setlength{\arrayrulewidth}{2pt}}\arrayrulecolor[HTML]{666666}\cline{1-2}\endhead



\multicolumn{1}{!{\color[HTML]{000000}\vrule width 0pt}>{\raggedright}p{\dimexpr 0.75in+0\tabcolsep+0\arrayrulewidth}}{\fontsize{11}{11}\selectfont{\textcolor[HTML]{000000}{\global\setmainfont{Arial}harden}}} & \multicolumn{1}{!{\color[HTML]{000000}\vrule width 0pt}>{\raggedright}p{\dimexpr 0.75in+0\tabcolsep+0\arrayrulewidth}!{\color[HTML]{000000}\vrule width 0pt}}{\fontsize{11}{11}\selectfont{\textcolor[HTML]{000000}{\global\setmainfont{Arial}Coming soon}}} \\





\multicolumn{1}{!{\color[HTML]{000000}\vrule width 0pt}>{\raggedright}p{\dimexpr 0.75in+0\tabcolsep+0\arrayrulewidth}}{\fontsize{11}{11}\selectfont{\textcolor[HTML]{000000}{\global\setmainfont{Arial}herstellen}}} & \multicolumn{1}{!{\color[HTML]{000000}\vrule width 0pt}>{\raggedright}p{\dimexpr 0.75in+0\tabcolsep+0\arrayrulewidth}!{\color[HTML]{000000}\vrule width 0pt}}{\fontsize{11}{11}\selectfont{\textcolor[HTML]{000000}{\global\setmainfont{Arial}Coming soon}}} \\

\noalign{\global\setlength{\arrayrulewidth}{2pt}}\arrayrulecolor[HTML]{666666}\cline{1-2}

\end{longtable}

\hypertarget{haken}{%
\subsubsection{Haken}\label{haken}}

\emph{haken} was presented with two transitive senses, two intransitive and one transitive/intransitive. The transitive senses are both concrete and literal and differ in specificity: one sense refers particularly to making somebody trip. Intransitive uses differ in literality and, while both might occur with \emph{blijven}, only the figurative definition mentions it (apparently restricting it). No figurative options are mentioned for the transitive senses, so if they occur annotators might either tag them as transitive concrete, intransitive figurative, or \emph{geen}. Finally, one sense that can occur as transitive or intransitve (ellided object) is that of `crochet'; it's so specific that it shouldn't be confused with others and would probably have high confidence.

\providecommand{\docline}[3]{\noalign{\global\setlength{\arrayrulewidth}{#1}}\arrayrulecolor[HTML]{#2}\cline{#3}}

\setlength{\tabcolsep}{2pt}

\renewcommand*{\arraystretch}{1.5}

\begin{longtable}[c]{|p{0.75in}|p{0.75in}}

\caption{Some caption.}\label{tab:haken}\\

\hhline{>{\arrayrulecolor[HTML]{666666}\global\arrayrulewidth=2pt}->{\arrayrulecolor[HTML]{666666}\global\arrayrulewidth=2pt}-}

\multicolumn{1}{!{\color[HTML]{000000}\vrule width 0pt}>{\raggedright}p{\dimexpr 0.75in+0\tabcolsep+0\arrayrulewidth}}{\fontsize{11}{11}\selectfont{\textcolor[HTML]{000000}{\global\setmainfont{Arial}lemma}}} & \multicolumn{1}{!{\color[HTML]{000000}\vrule width 0pt}>{\raggedright}p{\dimexpr 0.75in+0\tabcolsep+0\arrayrulewidth}!{\color[HTML]{000000}\vrule width 0pt}}{\fontsize{11}{11}\selectfont{\textcolor[HTML]{000000}{\global\setmainfont{Arial}text}}} \\

\noalign{\global\setlength{\arrayrulewidth}{2pt}}\arrayrulecolor[HTML]{666666}\cline{1-2}

\endfirsthead

\hhline{>{\arrayrulecolor[HTML]{666666}\global\arrayrulewidth=2pt}->{\arrayrulecolor[HTML]{666666}\global\arrayrulewidth=2pt}-}

\multicolumn{1}{!{\color[HTML]{000000}\vrule width 0pt}>{\raggedright}p{\dimexpr 0.75in+0\tabcolsep+0\arrayrulewidth}}{\fontsize{11}{11}\selectfont{\textcolor[HTML]{000000}{\global\setmainfont{Arial}lemma}}} & \multicolumn{1}{!{\color[HTML]{000000}\vrule width 0pt}>{\raggedright}p{\dimexpr 0.75in+0\tabcolsep+0\arrayrulewidth}!{\color[HTML]{000000}\vrule width 0pt}}{\fontsize{11}{11}\selectfont{\textcolor[HTML]{000000}{\global\setmainfont{Arial}text}}} \\

\noalign{\global\setlength{\arrayrulewidth}{2pt}}\arrayrulecolor[HTML]{666666}\cline{1-2}\endhead



\multicolumn{1}{!{\color[HTML]{000000}\vrule width 0pt}>{\raggedright}p{\dimexpr 0.75in+0\tabcolsep+0\arrayrulewidth}}{\fontsize{11}{11}\selectfont{\textcolor[HTML]{000000}{\global\setmainfont{Arial}haken}}} & \multicolumn{1}{!{\color[HTML]{000000}\vrule width 0pt}>{\raggedright}p{\dimexpr 0.75in+0\tabcolsep+0\arrayrulewidth}!{\color[HTML]{000000}\vrule width 0pt}}{\fontsize{11}{11}\selectfont{\textcolor[HTML]{000000}{\global\setmainfont{Arial}Coming soon}}} \\

\noalign{\global\setlength{\arrayrulewidth}{2pt}}\arrayrulecolor[HTML]{666666}\cline{1-2}

\end{longtable}

\hypertarget{expectations-for-the-annotation}{%
\section{Expectations for the annotation}\label{expectations-for-the-annotation}}

Here a first set of expectations for the annotations of the types has been summarized in six points, but discussion and revision are needed. They are formulated as predictions and followed by suggestions on how to confirm them. Some `technical' terms are:

\begin{itemize}
\tightlist
\item
  \textbf{majority sense}, meaning the sense tag that most of the annotators assigned to a given token.
\item
  \textbf{alternative sense/annotation}, meaning a sense tag assigned to a given token, different from the majority sense.
\item
  \textbf{(be) confuse(d)}, meaning there is disagreement on the annotation.
\end{itemize}

\begin{quote}
NOTE: It could also be useful to find some literature into this kind of annotations. I'm identifying the anthropocentrism of some sense distinctions as particularly foregrounded, but other than some considerations in the metaphor literature I don't really have theoretical backup. On the other hand, what I've seen of the `geen' annotations this far partially supports this intuition: some annotators would refuse to assign the \emph{horde 1} tag to cases of \emph{horde} + \emph{KTM's/vrachtwagens/danceprojecten/insecten}.
\end{quote}

\hypertarget{for-all-types}{%
\subsection{For all types}\label{for-all-types}}

\begin{enumerate}
\def\labelenumi{\arabic{enumi}.}
\tightlist
\item
  \textbf{Very specific senses will not be confused with more general senses}.
  When the majority sense is a very specific one, the only alternative annotation will be \emph{geen}.
  Concretely:
\end{enumerate}

\begin{itemize}
\tightlist
\item
  \emph{haken 5} and \emph{haken 3} will not be confused with each other nor with other senses.
\item
  \emph{heet 4} should not be confused with other senses.
\end{itemize}

\begin{enumerate}
\def\labelenumi{\arabic{enumi}.}
\setcounter{enumi}{1}
\tightlist
\item
  \textbf{Metaphor will be easier to identify than metonymy/specialization}
  If the metaphoric sense is an option distinct from the concrete/literal one, it won't often be confused with the literal counterpart; annotators will agree it's figurative. For metonymy and specialisation, there will be more disagreement and less confidence.
  Concretely:
\end{enumerate}

\begin{itemize}
\tightlist
\item
  \emph{spoor 1.1} and \emph{spoor 1.2} will be confused with each other more than with \emph{spoor 1.3} and also that more than with \emph{spoor 1.4}
\item
  \emph{blik 1.1} will be confused with \emph{1.2} more than with \emph{1.3}.
\item
  \emph{grijs 1} will be confused with \emph{2} and \emph{3} more than with \emph{6}.
\item
  \emph{goedkoop 1} will be confused with \emph{2} and \emph{3} more than with \emph{4}
\item
  adjectives with metaphoric distinctions (\emph{hoekig}, \emph{dof}, \emph{gekleurd}, \emph{heilzaam}) will present less confusion than those with metonymic distinctions (\emph{hachelijk}, \emph{hoopvol}, \emph{geestig})
\end{itemize}

\begin{enumerate}
\def\labelenumi{\arabic{enumi}.}
\setcounter{enumi}{2}
\tightlist
\item
  \textbf{Anthropocentric senses will be more easily distinguishable}.
  If the definition explicitly restricts the application to people, it won't be alternative annotation with other, non anthropo-exclusive senses. Borderline cases, due probably to unspecified context, would have low confidence.
  Concretely:
\end{enumerate}

\begin{itemize}
\tightlist
\item
  There should be low confusion in \emph{haten}, or at least low confidence in borderline cases
\item
  \emph{heet 5} should not be confused with others
\item
  \emph{grijs 4} should not be confused with other senses (except maybe 3, which is derived from it)
\item
  \emph{gekleurd 2} should not be confused with others
\item
  \emph{hoekig 3} should not be confused with others
\item
  \emph{dof 3} should not be confused with others
\item
  \emph{hoopvol} will present less confusion than \emph{geestig} and they both will present less confusion than \emph{hachelijk}
\end{itemize}

\hypertarget{only-for-nouns}{%
\subsection{Only for nouns}\label{only-for-nouns}}

\begin{enumerate}
\def\labelenumi{\arabic{enumi}.}
\setcounter{enumi}{3}
\tightlist
\item
  \textbf{Homonyms will not be confused with each other}.
  When a majority sense is from one homonym, alternative annotations will be of the same homonym or \emph{geen}.
\end{enumerate}

\hypertarget{only-for-verbs}{%
\section{Only for verbs}\label{only-for-verbs}}

The next two predictions overlap, and could explain different verb groupings (sometimes, different argument structure \emph{implies} subject distinction, but which is more prominent?).

\begin{enumerate}
\def\labelenumi{\arabic{enumi}.}
\setcounter{enumi}{4}
\tightlist
\item
  \textbf{Argument structure will be easier to identify than semantic differences.}
  Senses that differ in argument structure will not be confused with each other (won't be each other's alternatives) as much as senses with the same argument structure but different kinds of objects. The distinction will be probably easier to make with reflexive than with intransitive cases.
  Concretely: in general, transitive senses will be confused with each other but not with senses of a different argument structure (unless that sense is semantically very similar to that transitive sense). Cases that might generate confusion between different argument structures are:
\end{enumerate}

\begin{itemize}
\tightlist
\item
  gral. trans. \emph{haken} and fig.~intrans. \emph{haken} in cases of fig.~trans. \emph{haken}
\item
  fig.~trans. \emph{harden} and refl. \emph{harden}
\item
  fig./abs. trans. \emph{herstellen} and refl. \emph{herstellen}
\item
  spec. trans. \emph{herstructrureren} and intrans. \emph{herstructureren}
\item
  \emph{helpen 1} and intrans. \emph{helpen}
\end{itemize}

\begin{enumerate}
\def\labelenumi{\arabic{enumi}.}
\setcounter{enumi}{5}
\tightlist
\item
  \textbf{Senses that require different subjects will be easier to identify than senses that require different objects or prepositional arguments}
  Senses with different subject restrictions won't be each other alternatives. I'm thinking that subject restrictions are often linked to animacy, while object restriction might be more subtle in these cases.
  Concretely: in general, senses of any argument structure will not be confused with each other if one takes (mostly) animate subjects and the other one (mostly) inanimate subjects.
\end{enumerate}

\hypertarget{corpus}{%
\section{The corpus and samples}\label{corpus}}

The exploration of these samples of concordances also served for the calculation of the number of tokens we would have annotate. Regardless of the actual frequency of the items in the corpus, we extracted a minimum 240 tokens of each type (thinking of 6 batches of 40 tokens), and raised the amount to 280 if any of the senses had a relative frequency below 20\% in the sample, to 320 if it was below 10\%, and 360 if there were many senses and therefore some had a low frequency.

The sample of tokens was selected almost absolutely randomly. First all the instances of each type were extracted from the corpus; then, for each type as many \emph{files} as tokens we wanted to extract were selected, and from each file I randomly selected one token. Therefore, there qre no two instances of the same lemma from the same file in the samples.

The corpus is a selection of the LeNC and TwNC corpora, which include newspapers articles from Flanders and the Netherlands. This selection, performed by \href{@de_pascale_token-based_2019}{Stefano De Pascale} with an eye on a lectally balanced corpus, containes 4,614,267 types and 519,996,217 tokens (roughly 520M, 260 from Flanders and 260 from the Netherlands). The articles in the subcorpus were published between 1999 and 2004 in both quality and popular newspapers from both countries.

\hypertarget{annotation}{%
\section{Annotation procedure}\label{annotation}}

\hypertarget{assigning-batches-to-annotators}{%
\subsection{Assigning batches to annotators}\label{assigning-batches-to-annotators}}

\hypertarget{some-issues-because-we-are-humans}{%
\subsubsection{Some issues, because we are humans}\label{some-issues-because-we-are-humans}}

\hypertarget{annotation-toolinterface}{%
\subsection{Annotation tool/interface}\label{annotation-toolinterface}}

At the beginning of October 48 students from the General Linguistics course of the 2nd year of the Bachelor in Linguistics in KU Leuven were recruited to work as annotators of our tokens. Each of them was tasked with annotating 40 tokens of each of 12 types (at least three nouns, four adjectives and four verbs, plus one of either of the categories), a total of 480 tokens, for which we expect them to work an average of 10 hours, spread over 6 weeks. Students had the option of subscribing to double the number of tokens (and hours, and pay). Both the types and the sets of tokens were assigned randomly, while keeping in mind the part-of-speech distribution (the idea was to shuffle the tokens but for an issue in the code they were not, so different batches have tokens from different sources\ldots{} I noticed too late).

The annotation includes three compulsory pieces of information and one normally optional. For each of the tokens, they had to:

\begin{enumerate}
\def\labelenumi{\arabic{enumi}.}
\tightlist
\item
  assign a sense from a set of definitions/examples we will provide. If none of the senses is satisfactory, they may choose a ``None of the above'' options.
\item
  express the confidence of their decision in a Likert scale of 6 (illustrated as a star rating).
\item
  identify the words of the context (15 tokens to the left and to the right of the target, disregarding sentence boundaries but respecting those of the article) that helped them assign a sense.
\item
  if they couldn't assign a sense, they have to explain why. If they did assign one, they still have the option of adding extra information or thoughts on the annotation process, but it's not compulsory.
\end{enumerate}

Since entering textual information in a spreadsheet can easily lead to typos and inconsistencies and, furthermore, annotating the relevant context words (cues) is challenging with such system (you either have one row per contextword and decide a value for all the cases, or you select a reliable system for listing the context words, which couldn't be lemma or wordform but, better, position), we designed a user-friendly visual interface that transforms button-output in a json file with all the information required.

COMBAK Make the interface available again in its original form???
The \href{http://montesmariana.github.io/Annotation/}{interface} had a menu of types and, for the selected type, two tabs: an overview of the concordance lines and an annotation workspace. In the annotation workspace, they could read each line individually, click on the button corresponding to the sense they want to assign, rate their confidence with star rating, click on the words they found useful and enter any other comments. The overview section didn't only let them see the whole set of tokens to analyze, but the target items change color once they have been annotated and are themselves links to the Annotation tab for their concordance lines.

Since the site is a static webpage (a github page) and public, it doesn't gather information. It is indeed only an interface, and the users had to download the files with the data they have entered, in JSON format. They can also upload/open files from previous work, so they don't need to do everything in one go. Eventually, they had to send the JSON file to us.

The goals of the interface are twofold:

\begin{itemize}
\tightlist
\item
  it reduces typos and inconsistances for values that should be straightforward and present little variation (it was much faster to design the interface than it would have been to check the typos in 480 tokens times 40 annotators), and
\item
  it makes the annotation experience simpler and even more pleasant, letting the annotators focus their energies in the lexicographical task itself rather than in technicalities.
\end{itemize}

This is particularly evident for the task of selecting the relevant context words. With a spreadsheet, it would be either necessary to have separate rows per context words and annotate all of them (to make sure you are not forgetting any) or make a list of items in a cell of the row of the relevant concordance. That list would need to have something truly identifying of the context word, therefore not the form or the lemma but the position (since a same item could occur more than once in the context and not always have the same relevance), and counting them reliably would be time consuming and prone to errors. Clicking on the words so that the program itself lists the position of the relevant words, also making it visible which words were selected, solves both the easiness and reliability issues.

\hypertarget{known-issues}{%
\subsubsection{Known issues}\label{known-issues}}

The interface did have some issues, the consequences of which affect the output.

One technical issue was a bug in the code of the annotation, for which context words selected by an annotator might be replaced by other context words of the same wordform, but in a previous position. Once that bug was found, the annotators were warned, but not all of them necessarily checked their previous annotation very thoroughly. In any case, this only affects wordforms that occur more than once in the same concordance (which is not very often) and could be cleaned with some reasoning.

Another issue had to do with the format of the corpus, and could have been dealt with better. On the one hand, different sentences in the concordance were indiciated with a \VERB|\KeywordTok{\textless{}sentence\textgreater{}\textless{}/sentence\textgreater{}}| but had no impact in the rendering of the concordance, they were just replaced by empty spaces. They could've been replaced by \VERB|\KeywordTok{\textless{}p\textgreater{}\textless{}/p\textgreater{}}| tags. On the other hand, at some point of the corpus processing (before we had access to it), someone must have replaced all \emph{\&} with \emph{and}, so that HTML entities like \emph{\&quot;}, which would've translated into a \emph{"}, are rendered as \emph{andquot;}, which was extremely confusing for the annotators, especially in already complicated concordances full of them. This issue was identified too late (and in any case, if the corpus already reads it as \emph{andquot;} instead of \emph{"}, the confusion is for both).

\hypertarget{output}{%
\subsubsection{Output}\label{output}}

From each student, we received a file in json format where both their username and annotations where recorded. After checking that all tokens were annotated in all required variables (and that all `geen' cases had comments), the results were turned into tables and merged. After a number of attempts, we have two main tables:

First, a \href{C:/Users/u0118974/Box\%20Sync/Nederlands\%20wolken/Output/Merges/token_annotation.tsv}{register of tokens}, where each row is a token (id: \textbf{token\_id}) and the variables are only:
- \textbf{type}: the type that token belongs to;
- \textbf{batch}: the batch (set of 40 tokens) that token belongs to, named with the type plus a number;
- \textbf{majority\_sense}: the majority sense, or the tag that most of the annotators agreed on;
- When the tag was `geen' (``none of the above''), I classified the comments into kinds of justification that became alternative senses: \emph{between} (doubt between given alternatives), \emph{not\_listed} (suggestion of a different alternative from those given), \emph{unclear} (insufficient or confusing context, unknown words) and \emph{wrong\_lemma} (issues with lemmatization, part-of-speech tagging or even spelling, so that the concordance does not really correspond to the wanted target);
- when there was no agreement between two annotators (or three in the case of four annotations), the majority sense becomes \emph{no\_agreement};
- \textbf{majority\_agree}: the proportion (0-1) of annotators that voted for the majority sense;
- \textbf{majority\_conf}: the mean confidence (standardized by username-type combination) of the agreeing annotations;
- \textbf{mean\_conf}: the mean confidence (standardized by username-type combination) of all the annotations of the token.

Second, a \href{C:/Users/u0118974/Box\%20Sync/Nederlands\%20wolken/Output/Merges/long_tokens.tsv}{register of annotations}, where each row is a token-annotation combination (there is no row-id, but \textbf{token\_id} identifies the tokens and \textbf{annotator} identifies the annotations as \emph{ann\_1}, \emph{ann\_2}, \emph{ann\_3} or \emph{ann\_4}). For each row, the following variables are registered:
- \textbf{type}: the type that token belongs to;
- \textbf{batch}: the batch (set of 40 tokens) that token belongs to, named with the type plus a number;
- \textbf{username}: the name of the annotator (easier for me to keep track of, but nothing to publish);
- \textbf{code}: codename of the annotator, combining the number of `student' (set of batches of 12 different types) and number of annotator (matching the \textbf{annotator} variable);
- \textbf{annotators}: number of annotators who tagged the given token (normally 3). Not a very important variable;
- \textbf{original\_sense}: the sense tag applied by the annotator; either a sense represented by the name of the type and a number, or \emph{geen} for ``none of the above'';
- \textbf{confidence}: the confidence assigned by the annotator, with minimum 0 (1 star) and maximum 5 (6 stars);
- \textbf{conf\_std}: standardized confidence values, grouping by \textbf{username} and \textbf{type}\footnote{There is a lot of variation across annotators in how they used their confidence (even within the same set of tokens). There is also a wide variation depending on type, for each username. It also makes sense to use this criterion.};
- \textbf{comments}: the comments given by the annnotators, which are \emph{normally} empty, unless the sense is \emph{geen} (some annotators also commented tokens where they assigned an actual sense tag);
- \textbf{geen\_reason}: a classification of the comments given by the annotators, grouping them in the categories described before (\emph{between}, \emph{other\_sense}, \emph{unclear} and \emph{wrong\_lemma});
- \textbf{sense}: the modified sense annotation, replacing the \emph{geen} annotations of \textbf{original\_sense} with the categories from \textbf{geen\_reason};
- \textbf{annotations}: the number of different values of \textbf{sense} for that given token (so that 1 equals to total agreement);
- \textbf{agree\_nr}: the proportion (0-1) of annotators that assigned the \textbf{sense} of a given row to the \textbf{token\_id} of that row;
- \textbf{agree\_fct}: a categorical version of \textbf{agree\_nr}, where \emph{full} represents full agreement between annotators, \emph{none} no agreement, \emph{half} that two out of four agreed, \emph{minority} that the current row has a disgreeing annotation of the token and \emph{majority} if this annotation agrees with the majority for that token.
- There is no column with cues, but they are stored \emph{somewhere} so I can retrieve them when I want to start working on them.

\hypertarget{part-the-language-of-clouds}{%
\part{The language of clouds}\label{part-the-language-of-clouds}}

\hypertarget{nonsense-or-no-senses}{%
\chapter{Nonsense or no senses?}\label{nonsense-or-no-senses}}

In this part, which will have who knows how many chapters, I will delve into the
theroetico-methodological insights --the theoretical impact, if you will-- of
my analyses.

I will have to describe the annotation procedure (probably) but, more importantly,
discuss the main theoretical observations derived from my studies.

The main points, for now, are:

\begin{itemize}
\item
  From cloud interpretation, what we see are not necessarily senses, but rather
  ``common/shared/similar'' contexts of usage.
\item
  There is no single optimal solution for every item.
\item
  Because of the second point, what do the different parameter settings do,
  for specific items? (What kind of info do they pick up?)

  \begin{itemize}
  \tightlist
  \item
    This is also material for the first part of the project's monograph
  \end{itemize}
\end{itemize}

Currently (but there is still much research to do) I think that
different parameters offer different perspectives, but picking out different aspects
of the context.
Those perspectives won't be relevant for all lemmas --prepositions, if dependency-informed,
are relevant for \emph{hoop} but not for other nouns. Even that relevance is gradual.
We might be able to generalize, classifying items depending on which parameter settings
are relevant to them, or depending on what they would look like with certain or all kinds
of parameters.

Say, \emph{horde} and \emph{heffen} look quite good with just any setting, while \emph{haten} looks awful always.
Still, this is based on the current way of computing vectors,
which adds the type-level vectors of the tokens instead of averaging over them
(which is what I thought they did).
And I still have to look at the adjectives and revise nouns and verbs using the medoids based on
euclidean distances.

Will these analyses change my current conclusions? Chan chan chaaaannn\ldots{}

In any case, I will try to look at all of that in the last two weeks of February and then
discuss with DG the content for this part 😄.

\hypertarget{the-nature-of-clouds}{%
\chapter{The nature of clouds}\label{the-nature-of-clouds}}

When we talk about language, we often talk about the linguistic sign. That term itself
is most often linked to Saussure
and structuralism, but it is not far from the form-meaning pairing in cognitive linguistics.

The different branches and theoretical frameworks in linguistic differ by how they
define each of the elements of the pair. For example, referring to ``meaning'' or \emph{signifié}
aspect: is it linguistic internal? Does it include reference? Does it include \emph{social meaning}?
Moreover: should this relationship be studied in relation to an organized system,
or discovered from the turbulent chaos of language use?

Cognitive linguistics, the theoretical framework that has guided these studies, accepts
a broadly encompassing notion of \emph{meaning} ---which certainly does not make it easy to
define it--- and values the study of language in use. This is wonderfully compatible with
the Distributional Hypothesis, the methodological framework that inspires this research,
but can lead to disappointing misunderstandings.

The Distributional Hypothesis basically states a correlation between difference in
distributional behavior (difference in how forms are used) and semantic difference.

A popular interpretation of this Hypothesis involves using distributional patterns,
such as word vectors, as \emph{operationalizations of meaning}. The word space becomes a
semantic space, and a type-level model such as Latent Semantic Analysis (LSA)
``constructs a single vector for a word meaning'' \autocite[ 82]{bolognesi_2020}. This is the core
argument of these models, the selling point of distributional models and word embeddings.

However, as the analysis in this research project can make clear, \emph{usage is not meaning}.
Form can be simplified, but both usage and meaning are extremely complex phenomena,
and the simplification or abstraction we may make of one will not necessarily correlate
with the simplification or abstraction we prefer for the other. Usage correlates with
meaning, but that doesn't mean that any specific distributional model will correspond
to dictionary senses and the case is, unfortunately, that it doesn't.

Even \textcite{sahlgren_2006} makes a point of treating semantic similarity in very vague terms,
only distinguishing between paradigmatic and syntagmatic relationships.

A more useful idea might be to think of a triad of form, meaning, and usage. This
does not necessarily complexify the picture, since \emph{usage}, in its variety, is very
concrete and operationalizable. Even if we cannot equate usage patterns to meaning,
the relationship between form and usage patterns can still inform our understanding of
meaning. It certainly does not render distributional methods useless. We just have
to be careful not to conflate correlated but different phenomena, just like we should
not conflate form and meaning.

\printbibliography

\end{document}
